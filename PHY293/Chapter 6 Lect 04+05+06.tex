\documentclass[12pt]{article}
\usepackage{amsmath, amssymb, physics, siunitx, geometry}
\geometry{margin=1in}

\begin{document}

\title{\textbf{Chapter 6: Matter Waves}}
\author{Modern Physics Notes}
\date{}
\maketitle

\section*{Chapter Overview}
This chapter introduces the concept of \textbf{matter waves}, first proposed by Louis de Broglie in 1923, which unites the behaviour of particles and waves. Beginning with the de Broglie hypothesis, it establishes the idea that all material particles have wave-like properties characterized by a wavelength and frequency. The chapter explores experimental confirmation of this theory, the mathematical representation of quantum states via the wave function, and the implications of quantum uncertainty. The culmination is the development of the \textbf{Heisenberg Uncertainty Principle}, one of the foundational results of quantum mechanics.

\section{6.1 Introduction}
Bohr’s 1913 atomic model successfully explained the discrete spectral lines of hydrogen, but failed to explain:
\begin{itemize}
    \item Why angular momentum is quantized.
    \item The spectra of multi-electron atoms.
\end{itemize}

In 1923, \textbf{Louis de Broglie} proposed that all matter exhibits wave-like behaviour, introducing the concept of \textbf{matter waves}. His reasoning: if light behaves both as a wave and a particle, then matter might do the same. This symmetry between waves and particles bridged the Bohr model and modern quantum mechanics.

\section{6.2 De Broglie’s Hypothesis}
De Broglie proposed that particles obey the same relations as light:
\begin{equation}
    E = hf, \qquad p = \frac{h}{\lambda}
\end{equation}
where:
\begin{itemize}
    \item $E$ = energy of the particle,
    \item $h$ = Planck’s constant ($6.626\times10^{-34}\,\mathrm{J\cdot s}$),
    \item $f$ = frequency of the associated wave,
    \item $p$ = momentum of the particle,
    \item $\lambda$ = de Broglie wavelength.
\end{itemize}

Hence, for a non-relativistic particle of mass $m$ and velocity $v$:
\begin{equation}
    \boxed{\lambda = \frac{h}{mv}}
\end{equation}

De Broglie further argued that for an electron in a stable circular orbit:
\begin{equation}
    2\pi r = n\lambda, \qquad n = 1,2,3,\dots
\end{equation}
Substituting $\lambda = h/p$ gives
\begin{equation}
    rp = \frac{nh}{2\pi} \quad \Rightarrow \quad L = n\hbar
\end{equation}
Thus, Bohr’s quantization of angular momentum arises naturally from the standing wave condition.

\section{6.3 Experimental Verification}
Because de Broglie wavelengths are extremely small, their wave nature was difficult to observe.

\paragraph{Electron wavelength:}
\begin{equation}
    \lambda = \frac{h}{\sqrt{2mK}}
\end{equation}
where $K = \frac{1}{2}mv^2$ is kinetic energy.

\begin{center}
\begin{tabular}{c|c}
$K$ (eV) & $\lambda$ (nm) \\ \hline
10 & 0.39 \\
100 & 0.12 \\
1000 & 0.039 \\
10000 & 0.012
\end{tabular}
\end{center}

Electron wavelengths ($\sim 0.1$--$0.3$ nm) are comparable to X-ray wavelengths, enabling crystal diffraction.

\paragraph{Davisson–Germer Experiment (1927):}
\begin{itemize}
    \item 54 eV electrons scattered off a nickel crystal.
    \item Observed diffraction peaks consistent with Bragg’s Law:
    \begin{equation}
        n\lambda = 2d\sin\theta
    \end{equation}
    \item Verified de Broglie’s prediction $\lambda = h/p$.
\end{itemize}

\paragraph{G.P. Thomson (1927):}
\begin{itemize}
    \item Demonstrated electron diffraction through thin metal foils.
    \item Confirmed wave-like behaviour of electrons.
\end{itemize}

\section{The Quantum Wave Function (6.4)}

\subsection*{Complex Representation of Classical Waves}

For a classical 1D wave we often write
\begin{equation}
    y(x,t) = A \sin(kx - \omega t + \phi),
\end{equation}
but it is mathematically convenient to represent the same wave in complex form:
\begin{equation}
    y(x,t) = \Re\!\left[ A e^{\mathrm{i}(kx - \omega t)} \right].
\end{equation}
Using Euler's formula
\begin{equation}
    e^{\mathrm{i}\theta} = \cos\theta + \mathrm{i}\sin\theta,
\end{equation}
we can write
\begin{equation}
    A e^{\mathrm{i}(kx - \omega t)}
    = A\big[\cos(kx - \omega t) + \mathrm{i}\sin(kx - \omega t)\big].
\end{equation}
Thus the complex wave has a real and an imaginary part.  A general complex
quantity (and later, a wave function) can be written as
\begin{equation}
    \Psi = \Psi_{\text{real}} + \mathrm{i}\,\Psi_{\text{imag}},
\end{equation}
with magnitude
\begin{equation}
    |\Psi|
    = \sqrt{\Psi_{\text{real}}^{\,2} + \Psi_{\text{imag}}^{\,2}}.
\end{equation}

In \textbf{classical} waves:
\begin{itemize}
    \item The \emph{real part} represents the physical motion (e.g.\ displacement of the string, pressure variation in air, electric field).
    \item The \emph{imaginary part} is just a mathematical helper that makes differential equations and algebra easier.
    \item The magnitude $|\Psi|$ gives the (constant) amplitude of the wave.
    \item The complex form is therefore a \emph{tool}, not a directly physical quantity.
\end{itemize}

\subsection*{From Classical Waves to Matter Waves}

De~Broglie proposed that particles such as electrons also have wave-like
behavior. Their state can be represented by a wave function
\begin{equation}
    \Psi(x,t) = A e^{\mathrm{i}(kx - \omega t)}.
\end{equation}
Here:
\begin{itemize}
    \item $\Psi$ captures the \emph{wave aspect} of a particle --- a bridge between the particle and wave pictures.
    \item Both the real and imaginary parts are now physically meaningful: together they encode the wave-like nature of matter.
\end{itemize}

\subsection*{What Is a Wave Function?}

Every wave can be described mathematically by a wave function that gives the
disturbance at every point in space and time.
\begin{itemize}
    \item For classical waves (vibrating string, sound) the wave function describes a \emph{physical displacement} or field.
\end{itemize}

In \textbf{quantum mechanics}:
\begin{itemize}
    \item The wave function $\Psi$ is the \emph{complete mathematical description} of a quantum system.
    \item $\Psi$ encodes all measurable information: from it we can, in principle, extract position, momentum, energy, angular momentum, etc.
    \item Example: a particle in a 1D box has standing-wave solutions $\Psi_n(x)$; each allowed wave corresponds to a quantized energy $E_n$.
\end{itemize}

Unlike classical waves, $\Psi$ does not describe a literal oscillation of some
medium; it can even be complex-valued everywhere. This raises the central
interpretation questions that lead to Born’s proposal:
\begin{itemize}
    \item What does $\Psi$ \emph{actually} represent physically?
    \item Is $\Psi$ something physical (like displacement or pressure), or is it a different kind of quantity?
    \item Why must $\Psi$ be complex at all?
\end{itemize}

\subsection*{de Broglie’s extension to matter}
For matter waves, the state is described by a complex wave function $\Psi$:

\begin{equation}
    \Psi(x,t) = A e^{\mathrm{i}(kx - \omega t)}.
\label{eq:psiPlane}
\end{equation}
Unlike the classical case, both real and imaginary parts together encode physical information.

\subsection*{Born’s Interpretation (1926)}

Max Born proposed that the wave function $\Psi$ is not a physical oscillation of any
medium, but a \textbf{probability amplitude}.  
Its magnitude squared gives the probability density of finding the particle at a specific
position and time:
\begin{equation}
    P(\mathbf{r}, t) = |\Psi(\mathbf{r}, t)|^2 = \Psi^*(\mathbf{r}, t)\Psi(\mathbf{r}, t).
\end{equation}
Although $\Psi$ may be complex, $|\Psi|^2$ is always real, non-negative, and
experimentally measurable.

\paragraph{Interpretation.}
\begin{itemize}
    \item Larger $|\Psi|^2$ corresponds to a higher probability of detecting the particle there.
    \item The amplitude (magnitude) is $|\Psi|$, while the probability density is $|\Psi|^2$.
    \item The phase $\phi$ governs momentum, interference, and time evolution.
\end{itemize}

\paragraph{Amplitude–phase form.}
It is often convenient to express $\Psi$ as
\begin{equation}
    \Psi(x,t) = R(x,t)\, e^{i\phi(x,t)},
\end{equation}
where $R(x,t) = |\Psi(x,t)|$ is the amplitude and $\phi(x,t)$ is the phase.  
Quantum interference arises from the superposition of complex amplitudes:
interference occurs in $|\Psi|^2$, just as light intensity arises from interference of
electric-field amplitudes.

\vspace{0.3em}
\noindent\textit{Example:}  
In electron diffraction experiments, overlapping probability amplitudes interfere,
producing a measurable fringe pattern in $|\Psi|^2$ that mirrors optical interference.

% -------------------------------

\subsection*{Properties of the Wave Function}

If $|\Psi|^2$ gives the probability density, then $\Psi$ itself must obey specific
mathematical conditions to ensure physically meaningful results.

\paragraph{1. Single-valued, continuous, and finite.}
\begin{equation}
    \Psi(x,t) \text{ must have a single finite value at each } (x,t),
\end{equation}
changing smoothly without discontinuities.

\paragraph{2. Continuous first derivative.}
\begin{equation}
    \frac{d\Psi}{dx} \text{ must be continuous (except at infinite potential walls),}
\end{equation}
so that derived quantities such as momentum remain well-defined.

\paragraph{3. Normalizable.}
The total probability of finding the particle anywhere must equal 1:
\begin{equation}
    \int_{-\infty}^{\infty} |\Psi(x,t)|^2\,dx = 1.
\end{equation}

\vspace{0.3em}
\noindent
These properties ensure that $\Psi$ correctly represents a physical state.  
A non-normalizable or discontinuous function cannot correspond to a realizable particle. 

\subsection*{Connecting light and matter}
Classical intensity $\propto E^2$ (energy density); quantum intensity $\propto |\Psi|^2$ (probability density).
Interference in both cases follows from superposition, but quantum experiments measure \emph{probabilities}.

\subsection*{Double-slit meaning of $|\Psi|^2$}
Electrons sent one-by-one form interference fringes in the \emph{distribution} of detection points.
Attempting to measure ``which slit'' destroys interference (measurement disturbs the state).

\section{6.5 Which Slit Does the Electron Go Through?}
In the double-slit experiment, electrons fired one at a time through two slits form an interference pattern. 

Each electron strikes the screen as a particle, but the overall distribution is wave-like and consistent with $|\Psi|^2$

After many electrons:
\begin{itemize}
    \item Bright regions $\to$ large $|\Psi|^2$ (high probability)
    \item Dark regions $\to$ small $|\Psi|^2$ (low probability)
\end{itemize}

If detectors are placed to determine “which slit” the electron passes through, the measurement disturbs its momentum, destroying interference.

\textbf{Conclusion:} The act of measurement fundamentally alters the system; position and momentum cannot both be known precisely. Confirms that $|\Psi|$ predicts probabilistic outcomes, not deterministic paths. 


\section{6.6 Sinusoidal Waves}
A general sinusoidal (harmonic) wave describes oscillation in space and time:
\begin{equation}
    y(x,t) = A \sin(kx \mp \omega t + \phi)
\end{equation}
or equivalently, in complex form:
\begin{equation}
    \Psi(x,t) = A e^{i(kx \mp \omega t + \phi)}
\end{equation}


\paragraph{Definitions:}
\begin{align*}
    k &= \frac{2\pi}{\lambda} && \text{(wavenumber)}\\
    \omega &= 2\pi f && \text{(angular frequency)}\\
    v &= \frac{\omega}{k} = \lambda f && \text{(wave speed)}\\
    \phi &\text{ = initial phase}
\end{align*}

\noindent\text{Using de Broglie’s hypothesis, the sinusoidal form can describe light and matter waves.}

\begin{equation}
    E = \hbar \omega, \qquad 
    p = \frac{h}{\lambda} = \hbar k
\end{equation}

\begin{equation}
    \Psi(x,t) = A e^{i(kx \mp \omega t + \phi)} 
    = A e^{\frac{i}{\hbar}(px \mp Et + \phi)}
\end{equation}

\vspace{0.5em}
\noindent
\begin{center}
\renewcommand{\arraystretch}{1.4}
\begin{tabular}{p{6cm} p{6cm}}
\hline
\textbf{Classical Wave} & \textbf{Quantum (Matter) Wave} \\
\hline
$y(x,t) = A \sin(kx - \omega t + \phi)$ &
$\Psi(x,t) = A e^{i(kx - \omega t + \phi)}$ \\

$k = \dfrac{2\pi}{\lambda}$ &
$p = \hbar k$ \\

$\omega = 2\pi f$ &
$E = \hbar \omega$ \\

Photon: $E = hf = pc$ &
Particle: $E=\dfrac{p^2}{2m}$ (non-relativistic) \\
\hline
\end{tabular}
\end{center}

\vspace{0.8em}
\noindent\textbf{Takeaway:}\\
Light and matter share the same wave-equation form --- a unified description of wave--particle behaviour.

\section{6.7 Wave Packets and Fourier Analysis}

A perfectly sinusoidal wave has a single wavelength and frequency and extends infinitely in space and time.  
It cannot represent a localized particle because it would imply that the particle is equally likely to be found everywhere.  
Real particles are confined to a region of space, so their associated waves must also be localized.

\subsection*{Building Localized Waves}
A \textbf{wave packet} is a localized wave formed by combining many sinusoidal components of slightly different wavelengths (or wavenumbers).  
Mathematically, this process is called \textbf{Fourier analysis}.

\begin{equation}
    \Psi(x) = \int_{-\infty}^{\infty} A(k)\, e^{i kx}\, dk,
\end{equation}

\begin{itemize}
    \item $\Psi(x)$ describes the wave in \textbf{position space}.
    \item $A(k)$ is its \textbf{Fourier transform}, describing the wave in \textbf{wavenumber space} — how much of each $k$ component is present.
    \item The sharper (more localized) $\Psi(x)$ is in $x$, the broader its spread must be in $k$.
\end{itemize}

This relationship between $\Psi(x)$ and $A(k)$ connects the two most common representations of a quantum state — position and momentum — since $p = \hbar k$.

\subsection*{Fourier Series vs. Fourier Transform}
\begin{itemize}
    \item A \textbf{Fourier Series} represents a \emph{periodic} function as a discrete sum of sines or cosines:
    \[
        f(x) = \sum_{n=0}^{\infty} A_n \cos(k_n x),
        \quad k_n = \frac{2\pi n}{\lambda}.
    \]
    Higher $n$ values (shorter wavelengths) describe finer details.
    \item A \textbf{Fourier Transform} represents a \emph{nonperiodic} (localized) function as a continuous distribution of $k$:
    \[
        f(x) = \int_{-\infty}^{\infty} A(k)\, e^{i kx}\, dk, 
        \qquad 
        A(k) = \frac{1}{2\pi}\int_{-\infty}^{\infty} f(x)\, e^{-i kx}\, dx.
    \]
\end{itemize}

In quantum mechanics, the Fourier transform acts as a \textbf{change of basis} between position and momentum representations — a foundational idea of the Hilbert space formalism.

\subsection*{Localization and Uncertainty}
The trade-off between localization in $x$ and spread in $k$ arises naturally from Fourier theory:
\begin{equation}
    \boxed{\Delta x\, \Delta k \ge \frac{1}{2}}
\end{equation}
A smaller $\Delta x$ (more localized particle) requires a larger $\Delta k$ (broader wavenumber range).

\subsection*{Time–Frequency Analogy}
Exactly the same idea applies in time and frequency:
\begin{equation}
    \boxed{\Delta t\, \Delta \omega \ge \frac{1}{2}}
\end{equation}
A short pulse (small $\Delta t$) contains many frequencies (large $\Delta \omega$).  
This is the direct mathematical origin of the quantum uncertainty relations.

---

\section{6.8 The Uncertainty Relation for Position and Momentum}

From the de Broglie relation $p = \hbar k$, any spread in $k$ implies a spread in momentum:
\[
\Delta p = \hbar\,\Delta k.
\]
Combining with the Fourier relation $\Delta x\,\Delta k \ge 1/2$, we obtain the \textbf{Heisenberg uncertainty principle:}
\begin{equation}
    \boxed{\Delta x\, \Delta p \ge \frac{\hbar}{2}}
\end{equation}

\paragraph{Meaning.}  
This inequality states that a particle cannot simultaneously have perfectly known position and momentum.  
This is not due to experimental error — it reflects an \emph{intrinsic property of nature}: quantum objects simply do not possess exact values of both at the same time.

\paragraph{Consequences.}
\begin{itemize}
    \item A small $\Delta x$ (precisely localized particle) leads to a large $\Delta p$ (large momentum uncertainty).
    \item Conversely, a large $\Delta x$ (delocalized particle) gives a small $\Delta p$.
\end{itemize}

\paragraph{Examples.}
\begin{enumerate}
    \item \textbf{Macroscopic object:} A pellet of mass $10^{-5}\,$kg localized to $\Delta x = 0.5\,$mm gives $\Delta v \sim 10^{-23}\,$m/s — negligible.
    \item \textbf{Electron in an atom:} For $a = 0.1\,$nm, $\Delta v \approx 2.5\times10^{6}\,$m/s — comparable to actual electron speeds.
\end{enumerate}

\paragraph{Zero-Point Energy.}
Because a confined particle cannot have $\Delta p = 0$, it must possess a minimum kinetic energy:
\begin{equation}
    E_{\text{min}} \approx \frac{\hbar^2}{2m a^2}.
\end{equation}
This “zero-point energy” explains why particles can never be completely at rest inside a potential well.

\paragraph{Heisenberg Microscope.}
Any attempt to “see” an electron’s position requires light with wavelength comparable to $\Delta x$.  
But shorter wavelengths carry higher photon momentum, which disturbs the electron’s motion — increasing $\Delta p$.  
This trade-off leads naturally back to $\Delta x\,\Delta p \ge \hbar/2$.

---

\section{6.9 The Uncertainty Relation for Time and Energy}

The same Fourier trade-off applies to time and frequency, and using $E = \hbar\omega$ gives:
\begin{equation}
    \boxed{\Delta E\,\Delta t \ge \frac{\hbar}{2}}
\end{equation}
Here:
\begin{itemize}
    \item $\Delta E$ is the energy uncertainty of the state.
    \item $\Delta t$ is the characteristic lifetime or duration of that state.
\end{itemize}

\paragraph{Physical interpretation.}
\begin{itemize}
    \item Short-lived (transient) states have large energy uncertainty.
    \item Long-lived (stable) states have well-defined energies.
\end{itemize}

\paragraph{Example.}
An excited atomic state with lifetime $\Delta t = 10^{-8}$ s has
\[
\Delta E = \frac{\hbar}{2\Delta t} \approx 3\times10^{-8}\,\text{eV}.
\]
This small uncertainty causes \emph{natural line broadening} in emission spectra.

\paragraph{Stationary states.}
For $\Delta E = 0$, the particle’s energy is exact and $\Delta t \to \infty$ — the system remains forever in that energy state (a stationary state).

---

\section{6.10 Velocity of a Wave Packet}

For a free non-relativistic particle,
\[
E = \frac{p^2}{2m}.
\]
Using the de Broglie relations $E = \hbar\omega$ and $p = \hbar k$ gives
\[
\omega = \frac{\hbar k^2}{2m}.
\]

\subsection*{Phase and Group Velocity}
\begin{align}
v_{\text{phase}} &= \frac{\omega}{k} = \frac{E}{p} = \frac{p}{2m} = \frac{v_{\text{particle}}}{2}, \\
v_g &= \dv{\omega}{k} = \frac{p}{m} = v_{\text{particle}}.
\end{align}

\begin{itemize}
    \item \textbf{Phase velocity} is the speed of the individual wave crests — for matter waves it is \emph{half} the particle’s speed.
    \item \textbf{Group velocity} is the speed of the wave packet envelope — it equals the actual particle velocity.
\end{itemize}

\paragraph{Interpretation.}
A single sinusoidal matter wave does not represent a localized particle, but the \emph{group} of waves (the wave packet) moves at the same speed as the particle itself.

\section*{Summary of Key Relations}
\begin{align*}
    \lambda &= \frac{h}{p} 
        &\text{de Broglie wavelength}\\
    \lambda &= \frac{h}{\sqrt{2mK}} 
        &\text{electron wavelength vs.\ kinetic energy}\\
    2\pi r &= n\lambda 
        &\text{standing-wave / quantized orbit}\\
    n\lambda &= 2d\sin\theta 
        &\text{Bragg diffraction condition}\\[4pt]
    k &= \frac{2\pi}{\lambda}, \quad 
    \omega = 2\pi f 
        &\text{wave parameters}\\
    E &= hf = pc 
        &\text{photon energy--momentum}\\
    E &= \hbar\omega, \quad 
    p = \hbar k 
        &\text{quantum energy--momentum form}\\[4pt]
    P(\mathbf r,t) &= |\Psi(\mathbf r,t)|^2 
        &\text{probability density (Born)}\\
    \Psi(x,t) &= R(x,t)\,e^{i\phi(x,t)} 
        &\text{amplitude--phase form of }\Psi\\
    \int_{-\infty}^{\infty}|\Psi(x,t)|^2\,dx &= 1 
        &\text{normalization condition}\\[4pt]
    \Delta x\,\Delta k &\ge \tfrac{1}{2} 
        &\text{Fourier position--wavenumber}\\
    \Delta t\,\Delta \omega &\ge \tfrac{1}{2} 
        &\text{Fourier time--frequency}\\
    \Delta x\,\Delta p &\ge \tfrac{\hbar}{2} 
        &\text{Heisenberg position--momentum}\\
    \Delta E\,\Delta t &\ge \tfrac{\hbar}{2} 
        &\text{energy--time uncertainty}\\
    E_{\text{min}} &\approx \frac{\hbar^2}{2ma^2} 
        &\text{zero-point energy estimate}\\[4pt]
    E &= \frac{p^2}{2m}, \quad 
    \omega = \frac{\hbar k^2}{2m} 
        &\text{free-particle dispersion}\\
    v_{\text{phase}} &= \frac{\omega}{k} 
        = \frac{E}{p} = \frac{p}{2m} 
        &\text{phase velocity of matter wave}\\
    v_g &= \dv{\omega}{k} = \frac{p}{m} = v_{\text{particle}} 
        &\text{group (particle) velocity}
\end{align*}



\end{document}
