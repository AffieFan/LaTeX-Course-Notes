\documentclass[10pt]{article}
\usepackage[a4paper,margin=0.7in]{geometry}
\usepackage{amsmath,amssymb,bm}
\usepackage{enumitem}
\usepackage{titlesec}
\setlist[itemize]{noitemsep,topsep=0pt,leftmargin=1.2em}
\setlist[enumerate]{noitemsep,topsep=0pt,leftmargin=1.4em}
\renewcommand{\baselinestretch}{1.1}
\titleformat{\section}{\large\bfseries}{\thesection.}{0.5em}{}
\titleformat{\subsection}{\normalsize\bfseries}{\thesubsection}{0.5em}{}

\begin{document}
\begin{center}
{\LARGE \textbf{AER210 Midterm Field Guide — Multiple Integrals \& Vector Calculus}}\\[2mm]\end{center}

% ------------------------------
\section{Multiple Integrals}

\subsection{Regions \& Setup (Type I/II) — Stewart 15.1–15.2}
\textbf{Type I (vertical slices):}
\[
R=\{(x,y):a\le x\le b,\,g_1(x)\le y\le g_2(x)\},\quad
\iint_R f\,dA=\int_a^b\!\!\int_{g_1(x)}^{g_2(x)}f(x,y)\,dy\,dx
\]
\textbf{Type II (horizontal slices):}
\[
R=\{(x,y):c\le y\le d,\,h_1(y)\le x\le h_2(y)\},\quad
\iint_R f\,dA=\int_c^d\!\!\int_{h_1(y)}^{h_2(y)}f(x,y)\,dx\,dy
\]
\textbf{Midterm pattern:} reversing order/sketching region (2024 Q1b).  
\\
\textbf{Checklist:}
\begin{itemize}
\item Sketch axes, curves, intercepts.
\item Decide vertical/horizontal order.
\item Write new bounds before touching integrand.
\end{itemize}

\subsection{Double Integrals in Polar — Stewart 15.3}
Polar: \(x=r\cos\theta,\ y=r\sin\theta,\ dA=r\,dr\,d\theta.\)  
“Circle/annulus/rotational symmetry → polar.”  
Paraboloids intersection → footprint curve in \(xy\)‐plane gives \(r\)‐limits.
\[
V=\int_{\theta_1}^{\theta_2}\!\int_{r_1(\theta)}^{r_2(\theta)}
[z_{\text{top}}(r)-z_{\text{bot}}(r)]\,r\,dr\,d\theta
\]

\subsection{Applications of \(\iint\): Mass, COM, MOI — Stewart 15.4}
\[
m=\iint_R\rho\,dA,\qquad
\bar{x}=\frac{1}{m}\iint_R x\rho\,dA,\quad
\bar{y}=\frac{1}{m}\iint_R y\rho\,dA
\]
\[
I_O=\iint_R (x^2+y^2)\rho\,dA
\]
Change of variables often simplifies these (2022 Q6).

\subsection{Surface Area — Stewart 15.5 \& 16.6}
Graph \(z=f(x,y)\): \(dS=\sqrt{1+f_x^2+f_y^2}\,dA.\)  
Parametric \(\mathbf r(u,v)\): \(dS=\|\mathbf r_u\times\mathbf r_v\|\,du\,dv.\)
\begin{itemize}
\item Compute \(\mathbf r_u,\mathbf r_v\)
\item Cross product, then integrate magnitude over parameter box
\end{itemize}

\subsection{Triple Integrals — Stewart 15.6}
Switch order to simplify; choose suitable coordinates.  
Cylindrical: \(dV=r\,dr\,d\theta\,dz\); Spherical: \(dV=\rho^2\sin\phi\,d\rho\,d\phi\,d\theta.\)

\subsection{Cylindrical \& Spherical Coords — Stewart 15.7–15.8}
\[
\begin{aligned}
&\text{Cylindrical: } (r,\theta,z),\ x=r\cos\theta,\ y=r\sin\theta,\ dV=r\,dr\,d\theta\,dz\\
&\text{Spherical: } (\rho,\phi,\theta),\ x=\rho\sin\phi\cos\theta,\,y=\rho\sin\phi\sin\theta,\,z=\rho\cos\phi,\\
&dV=\rho^2\sin\phi\,d\rho\,d\phi\,d\theta
\end{aligned}
\]

\subsection{Taylor in Two Variables (2\textsuperscript{nd} order)}
\[
f(x,y)\approx f_0+f_x\Delta x+f_y\Delta y+\tfrac12(f_{xx}\Delta x^2+2f_{xy}\Delta x\Delta y+f_{yy}\Delta y^2)
\]
Used for local quadratic approximations; Hessian signs → shape info.

\subsection{Change of Variables \& Jacobian — Stewart 15.9}
\[
\iint_R f(x,y)\,dx\,dy=\iint_S f(x(u,v),y(u,v))
\Big|\frac{\partial(x,y)}{\partial(u,v)}\Big|\,du\,dv
\]
Choose new variables to straighten level curves.  
Checklist:
\begin{itemize}
\item Guess \(u,v\) from bounding curves.
\item Compute Jacobian \(J\).
\item Substitute and replace \(dx\,dy\) with \(|J|\,du\,dv.\)
\end{itemize}

% ------------------------------
\section{Vector Calculus}

\subsection{Line Integrals — Stewart 16.2}
Scalar: \(\displaystyle \int_C f\,ds=\int_a^b f(\mathbf r(t))\|\mathbf r'(t)\|dt\).  
Vector (work): \(\displaystyle \int_C \mathbf F\!\cdot\! d\mathbf r=\int_a^b \mathbf F(\mathbf r(t))\!\cdot\!\mathbf r'(t)\,dt.\)
\begin{itemize}
\item Parameterize \(C\); compute \(\mathbf r'(t)\).
\item Plug into correct formula.
\item If field looks like gradient, jump to FTLI.
\end{itemize}

\subsection{Fundamental Theorem for Line Integrals (FTLI) — Stewart 16.3}
If \(\mathbf F=\nabla f\), then \(\displaystyle \int_C\mathbf F\!\cdot\! d\mathbf r=f(B)-f(A)\).  
Conservative test: \(\nabla\times\mathbf F=\mathbf0\Rightarrow\) path independent.

\subsection{Green’s Theorem — Stewart 16.4}
\[
\oint_C P\,dx+Q\,dy=\iint_R\!\Big(\frac{\partial Q}{\partial x}-\frac{\partial P}{\partial y}\Big)dA
\]
Use for positively oriented closed \(C\).
\begin{itemize}
\item Check CCW orientation
\item Compute integrand
\item Integrate over region
\end{itemize}

\subsection{Parametric Surfaces \& Surface Area — Stewart 16.6}
\(\mathbf n=\mathbf r_u\times\mathbf r_v\), 
Area \(=\iint_D\|\mathbf r_u\times\mathbf r_v\|\,du\,dv.\)

\subsection{Surface Integrals — Stewart 16.7}
Scalar: \(\iint_S f\,dS\);\quad Flux: \(\iint_S \mathbf F\!\cdot\!\mathbf n\,dS.\)  
Graph \(z=f(x,y)\): 
\[
\mathbf n=\frac{\langle -f_x,-f_y,1\rangle}{\sqrt{1+f_x^2+f_y^2}},\quad 
\mathbf F\!\cdot\!\mathbf n\,dS=\mathbf F\!\cdot\!\langle -f_x,-f_y,1\rangle\,dx\,dy
\]

\subsection{Divergence \& Curl — Stewart 16.5}
\(\nabla\!\cdot\!\mathbf F\): sources/sinks;\quad \(\nabla\times\mathbf F\): rotation.  
\(\nabla\times(\nabla f)=\mathbf0.\)  
If curl = 0 on simply connected domain → conservative.

\subsection{Divergence Theorem — Stewart 16.9}
\[
\iint_{\partial V}\mathbf F\!\cdot\!\mathbf n\,dS=\iiint_V (\nabla\!\cdot\!\mathbf F)\,dV
\]

\subsection{Stokes’ Theorem — Stewart 16.8}
\[
\oint_{\partial S}\mathbf F\!\cdot\! d\mathbf r
=\iint_S (\nabla\times\mathbf F)\!\cdot\!\mathbf n\,dS
\]
Boundary orientation : right-hand rule.  
Checklist:
\begin{itemize}
\item Compute curl
\item Choose \(S\) with easy \(dS\)
\item Dot, convert to polar if disk footprint
\end{itemize}

% ------------------------------
\section{Past-Midterm Style Playbook}
\begin{itemize}
\item Reverse order → sketch and flip (2024 Q1b)
\item Volume between paraboloids → polar \(z_t-z_b\) (2022/2024)
\item Param curve line integral → param + plug (2024 Q2); if curl = 0 → FTLI
\item Green’s on polygon → double integral (2022 Q3; 2024 Q3)
\item Conservative → potential → \(f(B)-f(A)\)
\item Surface patch → param; flux via \(\mathbf r_u\times\mathbf r_v\)
\item Stokes on cap → curl, upward normal, polar disk (2024 Q8)
\item Change of vars → rectangularize (2022 Q6; 2024 Q6)
\end{itemize}

% ------------------------------
\section{Little Tricks \& Assumptions}
\begin{itemize}
\item Orientation matters (Green/Stokes/flux)
\item Always attach Jacobian: polar \(r\); cylindrical \(r\); spherical \(\rho^2\sin\phi\)
\item Conservative → closed loop integral = 0
\item Green’s prefers triangles/rectangles/disks
\item Switch order when integrand separates
\item Symmetry: odd -> 0; even -> double half-region
\item Piecewise curves: keep orientation or use Green’s
\end{itemize}

% ------------------------------
\section{Harder Integrals \& Trig Identities}

\subsection*{Basic Identities}
\[
\sin^2\theta + \cos^2\theta = 1
\]
\[
\int \sin\theta\cos\theta\,d\theta = \tfrac{1}{2}\sin^2\theta + C
\]

\subsection*{Power Reduction}
\[
\sin^2\theta = \tfrac{1 - \cos 2\theta}{2}, \qquad
\cos^2\theta = \tfrac{1 + \cos 2\theta}{2}
\]

\subsection*{Double-Angle Formulas}
\[
\begin{aligned}
\sin 2A &= 2\sin A\cos A,\\
\cos 2A &= \cos^2 A - \sin^2 A = 1 - 2\sin^2 A = 2\cos^2 A - 1,\\
\tan 2A &= \dfrac{2\tan A}{1 - \tan^2 A}
\end{aligned}
\]

\subsection*{Sum and Difference Formulas}
\[
\begin{aligned}
\sin(A \pm B) &= \sin A \cos B \pm \cos A \sin B,\\
\cos(A \pm B) &= \cos A \cos B \mp \sin A \sin B,\\
\tan(A \pm B) &= \dfrac{\tan A \pm \tan B}{1 \mp \tan A \tan B}
\end{aligned}
\]

\subsection*{Product-to-Sum Formulas}
\[
\begin{aligned}
\sin A \sin B &= \tfrac{1}{2}\big[\cos(A - B) - \cos(A + B)\big],\\
\cos A \cos B &= \tfrac{1}{2}\big[\cos(A - B) + \cos(A + B)\big],\\
\sin A \cos B &= \tfrac{1}{2}\big[\sin(A + B) + \sin(A - B)\big]
\end{aligned}
\]

\subsection*{Polar, Cylindrical, and Spherical Volume Elements}
\[
\begin{aligned}
dA &= r\,dr\,d\theta,\\
dV_{\text{cyl}} &= r\,dr\,d\theta\,dz,\\
dV_{\text{sph}} &= \rho^2\sin\phi\,d\rho\,d\phi\,d\theta
\end{aligned}
\]

% ------------------------------
\section{Quadric Surfaces (Recognition Cheat)}
\[
\begin{aligned}
&\text{Ellipsoid: } \frac{x^2}{a^2}+\frac{y^2}{b^2}+\frac{z^2}{c^2}=1\\
&\text{Cone: } \frac{x^2}{a^2}+\frac{y^2}{b^2}-\frac{z^2}{c^2}=0\\
&\text{Elliptic paraboloid: } \frac{x^2}{a^2}+\frac{y^2}{b^2}=\frac{z}{c}\\
&\text{Hyperboloid (1 sheet): } \frac{x^2}{a^2}+\frac{y^2}{b^2}-\frac{z^2}{c^2}=1\\
&\text{Hyperboloid (2 sheets): } -\frac{x^2}{a^2}-\frac{y^2}{b^2}+\frac{z^2}{c^2}=1\\
&\text{Hyperbolic paraboloid: } \frac{x^2}{a^2}-\frac{y^2}{b^2}=\frac{z}{c}
\end{aligned}
\]
Intersect with coordinate planes to identify quickly.

% ------------------------------
\section{One-Page Checklists}
A) Volume between two surfaces: intersect → polar → integrate \(r\).\\
B) Reverse order: sketch, rewrite as Type I/II.\\
C) Line integral: decide scalar/vector → FTLI if curl = 0.\\
D) Green’s: CCW → compute → integrate.\\
E) Conservative: curl 0 → find \(f\) → \(f(B)-f(A)\).\\
F) Param surface: \(\mathbf r_u\times\mathbf r_v\).\\
G) Stokes cap: upward normal → polar disk.\\
H) Change of vars: pick \(u,v\) → compute \(|J|\).

% ------------------------------
\section{Quick Definitions}
Type I/II regions: vertical vs horizontal.\\
Piecewise smooth curve: \(\mathbf r'(t)\) continuous/non-zero.\\
Conservative field: path independent; \(\mathbf F=\nabla f\).\\
Flux = flow across surface (\(\mathbf F\!\cdot\!\mathbf n\)).\\
Source/Sink : sign of divergence.\\
Laplacian \(\Delta f=\nabla\!\cdot\!\nabla f\).

% ------------------------------
\section{Mini Practice Prompts}
\begin{enumerate}
\item Reverse order of \(\int_0^1\!\!\int_x^1 e^{x/y}\,dy\,dx\)
\item Volume between \(z=4-r^2\) and \(z=3r^2\)
\item Check conservativity of \(\mathbf F=(3+2xy^2,2x^2y)\); find \(f\)
\item Use Green’s on triangle \((0,0),(2,1),(0,1)\)
\item Verify Stokes for paraboloid cap (upward normal)
\end{enumerate}

\end{document}
