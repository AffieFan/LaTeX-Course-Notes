\documentclass[11pt]{article}

\usepackage[a4paper,margin=1in]{geometry}
\usepackage{amsmath,amssymb}
\usepackage{physics}
\usepackage{siunitx}
\usepackage{graphicx}
\usepackage{caption}
\usepackage{subcaption}
\usepackage{hyperref}
\usepackage{enumitem}
\usepackage{float}

\setlength{\parskip}{0.5em}
\setlength{\parindent}{0pt}
\sisetup{per-mode=symbol}\documentclass[11pt]{article}

\usepackage[a4paper,margin=1in]{geometry}
\usepackage{amsmath,amssymb}
\usepackage{physics}
\usepackage{siunitx}
\usepackage{graphicx}
\usepackage{caption}
\usepackage{subcaption}
\usepackage{hyperref}
\usepackage{enumitem}
\usepackage{float}

\setlength{\parskip}{0.5em}
\setlength{\parindent}{0pt}
\sisetup{per-mode=symbol}


\title{Heat Transfer: Mechanisms and Steady Conduction\\
Lecture Notes }
\author{}
\date{}

\begin{document}
\maketitle

\tableofcontents

\newpage

%%%%%%%%%%%%%%%%%%%%%%%%%%%%%%%%%%%%%%%%%%%%%%%%%%%%%%%%%%%%
\section{Chapter 16: Mechanisms of Heat Transfer}
%%%%%%%%%%%%%%%%%%%%%%%%%%%%%%%%%%%%%%%%%%%%%%%%%%%%%%%%%%%%

\subsection{16.1 Introduction to Heat Transfer}

\textbf{Heat transfer} is the movement of energy due to a temperature difference.  
It is distinct from thermodynamics:

\begin{itemize}[nosep]
    \item \textbf{Thermodynamics:} Deals with initial and final equilibrium states.
    \item \textbf{Heat transfer:} Deals with the \emph{rate} and \emph{mechanism} of energy transfer
    between locations at different temperatures.
\end{itemize}

There are three basic mechanisms of heat transfer:
\begin{enumerate}[nosep]
    \item \textbf{Conduction} – is the transfer of heat
from the more energetic particles of a substance to the adjacent less energetic ones as a result of interactions between
the particles.
    \item \textbf{Convection} – is the transfer of heat between a solid surface and an adjacent moving fluid.
    \item \textbf{Thermal radiation} – energy transfer by electromagnetic waves due to the
    temperature of the bodies.
\end{enumerate}

In many practical situations, more than one mechanism acts simultaneously (e.g.\ hot object in
air loses heat by both convection and radiation).

%%%%%%%%%%%%%%%%%%%%%%%%%%%%%%%%%%%%%%%%%%%%%%%%%%%%%%%%%%%%
\subsection{16.2 Conduction and Fourier’s Law}

Consider a plane wall of thickness $\Delta x$ and area $A$, with its two faces held at
temperatures $T_1$ and $T_2$ $(T_1 > T_2)$.
Heat flows in the $x$-direction from the hot side to the cold side by conduction.

\begin{figure}[H]
    \centering
    \includegraphics[width=0.5\textwidth]{images/chap16_page2.png}
    \caption{Illustration of conduction through a plane wall.}
    \label{fig:chap16-conduction}
\end{figure}

\newpage

\subsubsection*{Macroscopic form of Fourier’s law}

The rate of heat transfer by conduction through the wall is

\begin{equation}
    \dot{Q}_{\text{cond}} = k A \frac{T_1 - T_2}{\Delta x}
    \label{eq:fourier-macro}
\end{equation}

\textbf{Where:}
\begin{itemize}[nosep]
    \item $\dot{Q}_{\text{cond}}$ = rate of heat transfer by conduction (\si{\watt})
    \item $k$ = thermal conductivity of the material (\si{\watt\per\meter\per\kelvin})
    \item $A$ = cross-sectional area normal to heat flow (\si{\meter\squared})
    \item $T_1, T_2$ = temperatures on the two faces of the wall (\si{\kelvin} or \si{\celsius})
    \item $\Delta x$ = wall thickness (\si{\meter})
\end{itemize}

\textbf{Purpose:} Eq.~\eqref{eq:fourier-macro} gives the \emph{overall rate} of heat conduction through
a plane layer when the temperature difference and thickness are known.

\subsubsection*{Differential form (one-dimensional steady conduction)}

In the limit as $\Delta x \to 0$, we obtain the differential form

\begin{equation}
    \dot{Q}_{\text{cond}} = -k A \frac{dT}{dx}
    \label{eq:fourier-diff}
\end{equation}

\textbf{Where:}
\begin{itemize}[nosep]
    \item $\frac{dT}{dx}$ = temperature gradient in the direction of heat flow (\si{\kelvin\per\meter}).
\end{itemize}

The minus sign indicates that heat flows in the direction of \emph{decreasing} temperature
(i.e.\ from hot to cold).

\textbf{Purpose:} Eq.~\eqref{eq:fourier-diff} is used when the temperature varies continuously
through the material and we want the local heat flux and differential equations for
temperature distribution.

\subsubsection*{Thermal conductivity}

\textbf{Thermal conductivity} $k$ is the rate of heat transfer through a unit thickness 
of the material per unit area per unit temperature difference (the ability of a material 
to conduct heat).

\begin{itemize}[nosep]
    \item High $k$ (e.g.\ metals) $\rightarrow$ good conductors.
    \item Low $k$ (e.g.\ foam, air) $\rightarrow$ good insulators.
\end{itemize}

Thermal conductivity generally depends on temperature and material structure
(pure metals, alloys, gases, liquids, etc.).

\paragraph{Microscopic interpretation.}
Temperature represents the \textbf{average kinetic energy} of the molecules or atoms of a substance.  
In liquids and gases, molecular motion includes translational, vibrational, and rotational components.  
When faster (hotter) and slower (colder) molecules collide, part of the kinetic energy of the faster
molecule is transferred to the slower one — much like elastic balls colliding at different velocities.
The greater the temperature difference, the more energetic these collisions, and thus the higher the rate
of energy transfer by conduction.

\paragraph{Kinetic theory of gases.}
From the kinetic theory, the thermal conductivity of gases is predicted — and experimentally confirmed — 
to vary as:

\begin{equation}
    k \propto \frac{\sqrt{T}}{\sqrt{M}}
    \label{eq:k-proportionality}
\end{equation}

\textbf{Where:}
\begin{itemize}[nosep]
    \item $T$ = thermodynamic temperature (\si{\kelvin})
    \item $M$ = molar mass of the gas (\si{\kilogram\per\mole})
\end{itemize}

Thus, for a particular gas (fixed $M$), $k$ increases with temperature because higher molecular speeds
increase the frequency of energy-exchanging collisions.  
At a fixed temperature, gases with higher molar mass $M$ conduct heat less effectively because heavier
molecules move more slowly.

\textbf{Example:}  
At $T = 1000~\si{\kelvin}$:
\begin{equation}
    k_{\text{He}} = 0.343~\si{\watt\per\meter\per\kelvin}, \quad
    k_{\text{air}} = 0.0667~\si{\watt\per\meter\per\kelvin}
    \label{eq:k-example}
\end{equation}
showing that helium ($M=4$) conducts heat much better than air ($M=29$).

\begin{figure}[H]
    \centering
    % Left figure
    \begin{minipage}[t]{0.3\textwidth}
        \centering
        \includegraphics[width=\linewidth]{images/Table16_1.png}
        \caption{Representative thermal conductivity values for various materials.}
        \label{fig:chap16-table}
    \end{minipage}
    \hfill
    % Right figure
    \begin{minipage}[t]{0.65\textwidth}
        \centering
        \includegraphics[width=\linewidth]{images/Figure 16_6.png}
        \caption{Range of thermal conductivity of various materials at room temperature.}
        \label{fig:chap16-range}
    \end{minipage}
\end{figure}

\subsubsection*{Thermal Diffusivity}

\textbf{Thermal diffusivity} $\alpha$ represents how quickly heat diffuses through a material.  
It indicates how fast a material responds to changes in temperature — that is, how fast heat 
spreads through it compared to how much energy it can store.

\begin{equation}
    \boxed{\alpha = \frac{k}{\rho c_p}} \quad [\si{\meter\squared\per\second}]
    \label{eq:thermal-diffusivity}
\end{equation}

\textbf{Where:}
\begin{itemize}[nosep]
    \item $\alpha$ = thermal diffusivity (\si{\meter\squared\per\second})
    \item $k$ = thermal conductivity (\si{\watt\per\meter\per\kelvin})
    \item $\rho$ = density (\si{\kilogram\per\meter\cubed})
    \item $c_p$ = specific heat capacity (\si{\joule\per\kilogram\per\kelvin})
\end{itemize}

\textbf{Concept:}
\begin{itemize}[nosep]
    \item $k$ measures how well a material \emph{conducts} heat.  
    \item $\rho c_p$ measures how much heat the material can \emph{store}.  
    \item Thus, $\alpha$ compares \emph{heat conduction} to \emph{heat storage}.
\end{itemize}

\textbf{Interpretation:}
\begin{itemize}[nosep]
    \item High $\alpha$ $\rightarrow$ heat diffuses quickly; the material reacts fast to temperature changes.
    \item Low $\alpha$ $\rightarrow$ heat moves slowly; the material absorbs heat and warms up gradually.
\end{itemize}

\subsubsection*{Heat flux}

The \textbf{heat flux} (heat transfer rate per unit area) is

\begin{equation}
    q_x'' = \frac{\dot{Q}_{\text{cond}}}{A} = -k \frac{dT}{dx}
    \label{eq:heat-flux}
\end{equation}

\textbf{Where:}
\begin{itemize}[nosep]
    \item $q_x''$ = heat flux in the $x$-direction (\si{\watt\per\meter\squared}).
\end{itemize}

\textbf{Purpose:} $q_x''$ measures the intensity of conduction at a surface or inside a material.

%%%%%%%%%%%%%%%%%%%%%%%%%%%%%%%%%%%%%%%%%%%%%%%%%%%%%%%%%%%%
\subsection{16.3 Convection}

\textbf{Convection} is heat transfer between a solid surface and a moving fluid in contact
with that surface. It combines:
\begin{itemize}[nosep]
    \item \textbf{Conduction} within the fluid layer adjacent to the surface, and
    \item \textbf{Bulk fluid motion} carrying energy away or toward the surface.
\end{itemize}

Convection is classified as:
\begin{itemize}[nosep]
    \item \textbf{Forced convection:} Fluid motion is caused by external means
    (fan, pump, wind, etc.).
    \item \textbf{Natural (free) convection:} Fluid motion is due to density differences
    caused by temperature gradients (buoyancy-driven).
\end{itemize}

\subsubsection*{Newton’s law of cooling}

The convective heat transfer rate from a surface at temperature $T_s$ to a fluid of
temperature $T_\infty$ is

\begin{equation}
    \dot{Q}_{\text{conv}} = h A_s (T_s - T_\infty)
    \label{eq:newton-cooling}
\end{equation}

\textbf{Where:}
\begin{itemize}[nosep]
    \item $\dot{Q}_{\text{conv}}$ = convective heat transfer rate (\si{\watt})
    \item $h$ = convection heat transfer coefficient (\si{\watt\per\meter\squared\per\kelvin})
    \item $A_s$ = surface area where convection occurs (\si{\meter\squared})
    \item $T_s$ = surface temperature (\si{\kelvin} or \si{\celsius})
    \item $T_\infty$ = temperature of the fluid far from the surface (\si{\kelvin} or \si{\celsius})
\end{itemize}

\textbf{Purpose:} Eq.~\eqref{eq:newton-cooling} is the basic correlation used to compute
convective heat loss or gain from surfaces when $h$ is known (typically from correlations
in later chapters).

\begin{figure}[h]
    \centering
    \includegraphics[width=0.5\textwidth]{images/chap16_page10.png}
    \caption{Illustration of convection from a surface to a moving fluid.}
    \label{fig:chap16-conv}
\end{figure}

%%%%%%%%%%%%%%%%%%%%%%%%%%%%%%%%%%%%%%%%%%%%%%%%%%%%%%%%%%%%
\subsection{16.4 Thermal Radiation}

\textbf{Thermal radiation} is the transfer of energy by electromagnetic waves (photons)
emitted by matter because of its temperature.

Key characteristics:
\begin{itemize}[nosep]
    \item Does not require a medium (can occur in a vacuum).
    \item Travels at the speed of light.
    \item Is characterized by wavelength and direction.
    \item All bodies at a temperature above absolute zero emit thermal radiation.
\end{itemize}

In heat transfer, we are primarily interested in \textbf{thermal radiation}, i.e.\ the portion
of electromagnetic radiation emitted due to the temperature of a body (not x-rays, radio waves,
etc.\ that may be unrelated to temperature).

Radiation is fundamentally a \textbf{volumetric} phenomenon (emission occurs within the medium),
but for opaque solids it is usually treated as a \textbf{surface} phenomenon, since radiation
emitted from deeper layers is absorbed within a few micrometres of the surface.

% --------------------------------------------------------
\subsubsection*{Blackbody and Stefan--Boltzmann law}

An ideal surface that emits the \emph{maximum possible} radiation at a given temperature is called a
\textbf{blackbody}. The radiation it emits is \textbf{blackbody radiation}.

The maximum rate of radiation that can be emitted from a surface of area $A_s$ at thermodynamic
temperature $T_s$ is given by the \textbf{Stefan--Boltzmann law}:
\begin{equation}
    \dot{Q}_{\text{emit,max}} = \sigma A_s T_s^4
    \label{eq:stefan-boltzmann}
\end{equation}

\textbf{Where:}
\begin{itemize}[nosep]
    \item $\dot{Q}_{\text{emit,max}}$ = maximum radiative heat emission rate from the surface (\si{\watt})
    \item $\sigma$ = Stefan--Boltzmann constant,
          $\sigma = 5.670\times 10^{-8}~\si{\watt\per\meter\squared\per\kelvin\tothe{4}}$
    \item $A_s$ = surface area (\si{\meter\squared})
    \item $T_s$ = surface absolute temperature (\si{\kelvin})
\end{itemize}

Sometimes it is convenient to define the \textbf{blackbody emissive power} (radiation per unit area):
\begin{equation}
    E_b = \sigma T_s^4
    \label{eq:blackbody-emissive}
\end{equation}
so that $\dot{Q}_{\text{emit,max}} = A_s E_b$.

\textbf{Purpose:} Eq.~\eqref{eq:stefan-boltzmann} gives the theoretical upper limit on radiation emission
from a surface at temperature $T_s$ (blackbody).

% --------------------------------------------------------
\subsubsection*{Real surfaces and emissivity}

Real surfaces emit less radiation than a blackbody at the same temperature.  
This deviation is quantified by the \textbf{emissivity} $\varepsilon$:
\begin{equation}
    \dot{Q}_{\text{emit}} = \varepsilon \sigma A_s T_s^4
    \label{eq:real-surface-emission}
\end{equation}

\textbf{Where:}
\begin{itemize}[nosep]
    \item $\dot{Q}_{\text{emit}}$ = radiative heat emission rate from the real surface (\si{\watt})
    \item $\varepsilon$ = emissivity of the surface, $0 \le \varepsilon \le 1$ (dimensionless)
\end{itemize}

An ideal blackbody has $\varepsilon = 1$.  
Typical emissivities for common materials at $T \approx 300~\si{\kelvin}$ are listed in Table 16--6.

\textbf{Interpretation:}
\begin{itemize}[nosep]
    \item $\varepsilon$ close to 1 $\rightarrow$ surface is a good emitter and good absorber.
    \item $\varepsilon$ small $\rightarrow$ surface is a poor emitter/absorber (often a good reflector).
\end{itemize}

% --------------------------------------------------------
\subsubsection*{Absorptivity and Kirchhoff’s law}

In addition to emitting radiation, a surface can \textbf{absorb} or \textbf{reflect}
incident radiation. For opaque surfaces (no transmission),
\begin{equation}
    \alpha + \rho = 1
    \label{eq:alpha-plus-rho}
\end{equation}
where
\begin{itemize}[nosep]
    \item $\alpha$ = \textbf{absorptivity}: fraction of incident radiation absorbed,
    \item $\rho$ = reflectivity: fraction reflected.
\end{itemize}

The rate at which a surface absorbs radiation is
\begin{equation}
    \dot{Q}_{\text{absorbed}} = \alpha \,\dot{Q}_{\text{incident}}
    \label{eq:absorbed}
\end{equation}

\textbf{Where:}
\begin{itemize}[nosep]
    \item $\dot{Q}_{\text{absorbed}}$ = rate of radiation energy absorbed by the surface (\si{\watt})
    \item $\dot{Q}_{\text{incident}}$ = rate of incident radiation on the surface (\si{\watt})
    \item $\alpha$ = absorptivity (dimensionless, $0 \le \alpha \le 1$)
\end{itemize}

A blackbody has $\alpha = 1$ and absorbs all incident radiation.

\paragraph{Kirchhoff’s law of radiation.}
For a surface in thermal equilibrium at a given temperature and wavelength,
\begin{equation}
    \varepsilon = \alpha
    \label{eq:kirchhoff}
\end{equation}
That is, \emph{good emitters are also good absorbers} (and poor emitters are poor absorbers).

% --------------------------------------------------------
\subsubsection*{Net radiation exchange with large surroundings}

Consider a small surface ($A_s$) of emissivity $\varepsilon$ at temperature $T_s$, completely
enclosed by a much larger isothermal surface at temperature $T_{\text{sur}}$ that behaves as a blackbody.
The net radiation heat transfer from the small surface to the surroundings is:

\begin{equation}
    \dot{Q}_{\text{rad}} = \varepsilon \sigma A_s \left( T_s^4 - T_{\text{sur}}^4 \right)
    \label{eq:net-rad}
\end{equation}

\textbf{Where:}
\begin{itemize}[nosep]
    \item $\dot{Q}_{\text{rad}}$ = net radiative heat transfer from the surface (\si{\watt})
    \item $T_{\text{sur}}$ = absolute temperature of the surrounding surfaces (\si{\kelvin})
\end{itemize}

\textbf{Purpose:} Eq.~\eqref{eq:net-rad} is used when a surface exchanges radiation with
large, approximately isothermal surroundings (e.g.\ a small object in a large room or the sky).

% --------------------------------------------------------
\subsubsection*{Combined convection and radiation}

In many practical situations (e.g.\ a hot surface in air), heat transfer from a surface occurs
\emph{simultaneously} by convection and radiation.

The total heat transfer rate is obtained by adding the convective and radiative contributions:
\begin{equation}
    \dot{Q}_{\text{total}} = \dot{Q}_{\text{conv}} + \dot{Q}_{\text{rad}}
    \label{eq:qtotal-sum}
\end{equation}

With
\begin{equation}
    \dot{Q}_{\text{conv}} = h_{\text{conv}} A_s (T_s - T_\infty),
    \qquad
    \dot{Q}_{\text{rad}} = \varepsilon \sigma A_s (T_s^4 - T_{\text{sur}}^4),
    \label{eq:qconv-qrad-def}
\end{equation}
we can define a \textbf{combined heat transfer coefficient} $h_{\text{combined}}$ such that
\begin{equation}
    \dot{Q}_{\text{total}} = h_{\text{combined}} A_s (T_s - T_\infty)
    \label{eq:qtotal-combined}
\end{equation}

For many engineering applications it is reasonable to take $T_{\text{sur}} \approx T_\infty$,
which leads to
\begin{equation}
    h_{\text{combined}} = h_{\text{conv}} + h_{\text{rad}}
    \label{eq:hcombined-def}
\end{equation}
with the \textbf{radiation heat transfer coefficient}
\begin{equation}
    h_{\text{rad}} = \varepsilon \sigma (T_s + T_{\text{sur}})(T_s^2 + T_{\text{sur}}^2)
    \label{eq:hrad}
\end{equation}

\textbf{Where:}
\begin{itemize}[nosep]
    \item $h_{\text{conv}}$ = convection heat transfer coefficient (\si{\watt\per\meter\squared\per\kelvin})
    \item $h_{\text{rad}}$ = equivalent radiation heat transfer coefficient (\si{\watt\per\meter\squared\per\kelvin})
    \item $h_{\text{combined}}$ = effective combined convection–radiation coefficient (\si{\watt\per\meter\squared\per\kelvin})
    \item $T_\infty$ = fluid (air) temperature far from the surface (\si{\kelvin})
\end{itemize}

\textbf{Purpose:} Using $h_{\text{combined}}$ allows us to treat convection and radiation together as one
effective surface heat transfer mechanism when computing overall energy balances.

%%%%%%%%%%%%%%%%%%%%%%%%%%%%%%%%%%%%%%%%%%%%%%%%%%%%%%%%%%%%
\subsection{16.5 Simultaneous Heat Transfer Mechanisms}

Although there are three fundamental mechanisms of heat transfer --- \textbf{conduction},
\textbf{convection}, and \textbf{radiation} --- not all three occur simultaneously in every medium.
Depending on the material and environment, one or two modes may dominate.
\begin{center}
\textbf{Summary of Heat Transfer Modes by Medium}
\end{center}
\begin{table}[H]
\centering
\renewcommand{\arraystretch}{1.2}
\begin{tabular}{|p{0.22\textwidth}|p{0.7\textwidth}|}
\hline
\textbf{Medium} & \textbf{Dominant Mechanisms of Heat Transfer} \\ \hline
\textbf{Opaque Solid} &
Heat transfer occurs by \textbf{conduction only} within the solid since radiation cannot penetrate
deeply. The surface may also exchange heat by convection and radiation with its surroundings. \\ \hline
\textbf{Semitransparent Solid} &
Both \textbf{conduction and radiation} occur inside the material (e.g., glass, some ceramics). \\ \hline
\textbf{Still Fluid (no motion)} &
Heat transfer occurs by \textbf{conduction} and possibly \textbf{radiation}. \\ \hline
\textbf{Flowing Fluid} &
Heat transfer occurs by \textbf{convection} (combined conduction + bulk fluid motion)
and \textbf{radiation}. \\ \hline
\textbf{Vacuum} &
Only \textbf{radiation} can occur since conduction and convection require a material medium. \\ \hline
\end{tabular}
\end{table}

\paragraph{Examples:}
\begin{itemize}[nosep]
    \item A cold rock in warm air gains heat at its surface by \textbf{convection} (air) and
    \textbf{radiation} (from surroundings), and transfers it inward by \textbf{conduction}.
    \item Inside a gas, heat transfer is either by \textbf{conduction} (no motion) or
    \textbf{convection} (with bulk motion) — not both.
    \item Gases are mostly \textbf{transparent to radiation}, except certain gases like ozone,
    which absorb specific wavelengths (e.g., ultraviolet).
    \item Liquids are typically \textbf{strong absorbers} of radiation.
\end{itemize}

\paragraph{Key Takeaways:}
\begin{itemize}[nosep]
    \item Conduction dominates in solids; convection dominates in fluids.
    \item Radiation becomes significant when temperature differences are large
    or when a vacuum is present.
    \item In most systems, two modes act together (e.g., convection + radiation from a hot surface).
\end{itemize}

\begin{figure}[H]
    \centering
    \begin{minipage}[c]{0.18\textwidth}
        \centering
        \fbox{%
            \includegraphics[width=\linewidth]{images/chap16_page15.png}%
        }
    \end{minipage}%
    \hfill
    \begin{minipage}[c]{0.75\textwidth}
        \captionof{figure}{Typical combinations of heat transfer modes in solids, fluids, and vacuum.}
        \label{fig:chap16-simultaneous}
    \end{minipage}
\end{figure}

%%%%%%%%%%%%%%%%%%%%%%%%%%%%%%%%%%%%%%%%%%%%%%%%%%%%%%%%%%%%
\section{Chapter 17: Steady Heat Conduction}

%%%%%%%%%%%%%%%%%%%%%%%%%%%%%%%%%%%%%%%%%%%%%%%%%%%%%%%%%%%%
\subsection{17.1 Steady Heat Conduction in Plane Walls}

We analyze one-dimensional, steady conduction in a plane wall of thickness $L$, with
no heat generation, and constant properties.

Assumptions:
\begin{itemize}[nosep]
    \item Steady state ($\partial T / \partial t = 0$).
    \item One-dimensional conduction in the $x$-direction.
    \item No internal heat generation.
    \item Constant thermal conductivity $k$.
\end{itemize}

\subsubsection*{Temperature distribution}

The governing differential equation for 1-D steady conduction without heat generation is

\begin{equation}
    \frac{d^2 T}{dx^2} = 0
    \label{eq:wall-governing}
\end{equation}

Integrating twice:
\[
    T(x) = C_1 x + C_2
\]
Constants are found from boundary conditions:
\[
    T(0) = T_1, \quad T(L) = T_2
\]
which yields a linear temperature distribution

\begin{equation}
    T(x) = T_1 + (T_2 - T_1)\frac{x}{L}
    \label{eq:temp-plane-wall}
\end{equation}

\textbf{Where:}
\begin{itemize}[nosep]
    \item $T(x)$ = temperature at position $x$ in the wall (\si{\kelvin})
    \item $T_1$ = temperature at $x=0$ (\si{\kelvin})
    \item $T_2$ = temperature at $x=L$ (\si{\kelvin})
    \item $L$ = wall thickness (\si{\meter})
\end{itemize}

\textbf{Purpose:} Eq.~\eqref{eq:temp-plane-wall} gives the steady-state temperature
profile through a plane wall under these conditions.

\subsubsection*{Heat transfer rate and thermal resistance of a plane layer}

Using Fourier’s law, for steady state the heat transfer rate is constant:

\begin{equation}
    \dot{Q} = k A \frac{T_1 - T_2}{L}
    \label{eq:plane-wall-Q}
\end{equation}

This can be written in a \textbf{resistance form}:

\begin{equation}
    \dot{Q} = \frac{T_1 - T_2}{R_{\text{cond}}}
    \quad\text{with}\quad
    R_{\text{cond}} = \frac{L}{k A}
    \label{eq:plane-wall-resistance}
\end{equation}

\textbf{Where:}
\begin{itemize}[nosep]
    \item $R_{\text{cond}}$ = thermal resistance of the wall layer (\si{\kelvin\per\watt})
\end{itemize}

\textbf{Purpose:} This form is analogous to electrical circuits and allows us to build
thermal resistance networks.

%%%%%%%%%%%%%%%%%%%%%%%%%%%%%%%%%%%%%%%%%%%%%%%%%%%%%%%%%%%%
\subsection{17.2 Convection Resistance and Basic Thermal Resistance Network}

For convection at a surface of area $A_s$, Newton’s law of cooling can be written as

\begin{equation}
    \dot{Q} = h A_s (T_s - T_\infty)
    = \frac{T_s - T_\infty}{R_{\text{conv}}}
    \quad\text{with}\quad
    R_{\text{conv}} = \frac{1}{h A_s}
    \label{eq:conv-resistance}
\end{equation}

\textbf{Where:}
\begin{itemize}[nosep]
    \item $R_{\text{conv}}$ = convection thermal resistance (\si{\kelvin\per\watt})
    \item $T_s$ = surface temperature (\si{\kelvin})
    \item $T_\infty$ = fluid temperature far from surface (\si{\kelvin})
\end{itemize}

\textbf{Purpose:} Convection is modeled as a surface resistance, analogous to conduction
resistance, enabling combined conduction--convection networks.

\subsubsection*{Wall between two fluids: series resistances}

For a plane wall of thickness $L$ separating two fluids at temperatures $T_{\infty 1}$ and $T_{\infty 2}$:

\begin{itemize}[nosep]
    \item Convection from hot fluid to surface 1: $R_{\text{conv},1} = 1 / (h_1 A)$
    \item Conduction through the wall: $R_{\text{cond}} = L / (k A)$
    \item Convection from surface 2 to cold fluid: $R_{\text{conv},2} = 1 / (h_2 A)$
\end{itemize}

Total resistance (in series) is

\begin{equation}
    R_{\text{total}} = R_{\text{conv},1} + R_{\text{cond}} + R_{\text{conv},2}
    \label{eq:series-resistance}
\end{equation}

and
\begin{equation}
    \dot{Q} = \frac{T_{\infty 1} - T_{\infty 2}}{R_{\text{total}}}
    \label{eq:Q-series}
\end{equation}

\textbf{Purpose:} Eqs.~\eqref{eq:series-resistance} and \eqref{eq:Q-series} provide a compact
way to compute the heat transfer rate through composite plane systems.

\subsubsection*{Radiation resistance and combined convection--radiation}

A surface exposed to a gas usually exchanges heat by both \textbf{convection} and \textbf{thermal radiation}.

Net radiation heat transfer between a surface (area $A_s$, temperature $T_s$) and large surroundings at $T_{\text{surr}}$ can be written as

\begin{equation}
    \dot{Q}_{\text{rad}} = \varepsilon \sigma A_s \bigl( T_s^4 - T_{\text{surr}}^4 \bigr)
    = h_{\text{rad}} A_s (T_s - T_{\text{surr}})
    = \frac{T_s - T_{\text{surr}}}{R_{\text{rad}}}
    \label{eq:rad-resistance}
\end{equation}

\textbf{Where:}
\begin{itemize}[nosep]
    \item $\varepsilon$ = surface emissivity (dimensionless),
    \item $\sigma$ = Stefan--Boltzmann constant ($5.67\times 10^{-8}\,\si{\watt\per\meter\squared\per\kelvin\tothe4}$),
    \item $h_{\text{rad}}$ = \emph{radiation} heat transfer coefficient (\si{\watt\per\meter\squared\per\kelvin}),
    \item $R_{\text{rad}} = 1/(h_{\text{rad}} A_s)$ = radiation resistance.
\end{itemize}

An effective radiation coefficient is defined by linearizing:

\begin{equation}
    h_{\text{rad}} = \varepsilon \sigma \bigl( T_s^2 + T_{\text{surr}}^2 \bigr)(T_s + T_{\text{surr}})
    \label{eq:hrad-def}
\end{equation}

When the surrounding air and surfaces are at about the same temperature, $T_{\text{surr}} \approx T_\infty$, radiation and convection from the same surface may be lumped into a \textbf{combined coefficient}:

\begin{equation}
    h_{\text{combined}} = h_{\text{conv}} + h_{\text{rad}}
    \quad\Rightarrow\quad
    \dot{Q} = h_{\text{combined}} A_s (T_s - T_\infty)
    \label{eq:h-combined}
\end{equation}

\textbf{Purpose:} This lets us treat convection + radiation as a single surface resistance,
\[
    R_{\text{surf,combined}} = \frac{1}{h_{\text{combined}} A_s},
\]
so the resistance network stays simple (only ``convection-like'' nodes).

\begin{figure}[h]
    \centering
    \includegraphics[width=0.5\textwidth]{images/thermalcontactconductance.png}
    \caption{Plane wall conduction and temperature profile (Chapter 17).}
    \label{fig:chap17-plane}
\end{figure}

%%%%%%%%%%%%%%%%%%%%%%%%%%%%%%%%%%%%%%%%%%%%%%%%%%%%%%%%%%%%
\subsection{17.3 Composite Walls and Generalized Resistance Networks}

For multiple layers in series (each with thickness $L_i$, conductivity $k_i$, and area $A$):

\begin{equation}
    R_{\text{cond,total}} = \sum_{i} \frac{L_i}{k_i A}
    \label{eq:composite-wall-R}
\end{equation}

Total resistance for a system with convection on both sides becomes:

\begin{equation}
    R_{\text{total}} =
    \frac{1}{h_1 A}
    + \sum_i \frac{L_i}{k_i A}
    + \frac{1}{h_2 A}
    \label{eq:composite-total-R}
\end{equation}

and the steady heat transfer rate is

\begin{equation}
    \dot{Q} = \frac{T_{\infty 1} - T_{\infty 2}}{R_{\text{total}}}
    \label{eq:Q-composite}
\end{equation}

\textbf{Purpose:} These equations generalize the thermal resistance idea to
multi-layer systems.

More complex systems may have \textbf{parallel thermal paths} (analogous to parallel
resistors in circuits). For such cases:

\begin{equation}
    \frac{1}{R_{\text{eq}}} = \sum_j \frac{1}{R_j}
    \label{eq:parallel-R}
\end{equation}

\textbf{Where:}
\begin{itemize}[nosep]
    \item $R_j$ = individual thermal resistances for each parallel path,
    \item $R_{\text{eq}}$ = equivalent resistance seen between the same two nodes.
\end{itemize}

\textbf{Purpose:} Parallel networks are used when heat can flow through more than one
path simultaneously (e.g.\ through a stud and insulation in a wall).

%%%%%%%%%%%%%%%%%%%%%%%%%%%%%%%%%%%%%%%%%%%%%%%%%%%%%%%%%%%%
\subsection{17.4 Thermal Contact Resistance}

At the interface between two solids, microscopic roughness leads to actual
contact only at small spots, with thin gaps filled with air or other fluid.
This causes an additional \textbf{thermal contact resistance} between the solids.

The heat transfer across the interface can be expressed as

\begin{equation}
    \dot{Q} = h_c A (T_{s1} - T_{s2})
    = \frac{T_{s1} - T_{s2}}{R_c}
    \quad\text{with}\quad
    R_c = \frac{1}{h_c A}
    \label{eq:contact-resistance}
\end{equation}

\textbf{Where:}
\begin{itemize}[nosep]
    \item $h_c$ = thermal contact conductance (\si{\watt\per\meter\squared\per\kelvin})
    \item $A$ = apparent contact area (\si{\meter\squared})
    \item $T_{s1}$, $T_{s2}$ = temperatures of the two solid surfaces at the interface (\si{\kelvin})
    \item $R_c$ = thermal contact resistance (\si{\kelvin\per\watt})
\end{itemize}

\textbf{Purpose:} Eq.~\eqref{eq:contact-resistance} allows us to model the imperfect
heat transfer at solid--solid interfaces as an extra resistance in series.

Thermal contact resistance depends on:
\begin{itemize}[nosep]
    \item Contact pressure,
    \item Surface roughness and hardness,
    \item Interstitial material (air, grease, filler, etc.).
\end{itemize}

%%%%%%%%%%%%%%%%%%%%%%%%%%%%%%%%%%%%%%%%%%%%%%%%%%%%%%%%%%%%
\subsection{17.5 Heat Conduction in Cylinders and Spheres}

For cylindrical or spherical geometries with radial conduction, the area normal to
heat flow varies with radius, so the thermal resistance takes a different form.

\subsubsection*{Radial conduction in a cylinder}

For a hollow cylinder with inner radius $r_1$, outer radius $r_2$, length $L$,
and constant conductivity $k$, under 1-D steady radial conduction and no heat generation:

\begin{equation}
    \dot{Q} = \frac{2\pi k L (T_1 - T_2)}{\ln(r_2 / r_1)}
    \label{eq:cylinder-Q}
\end{equation}

\textbf{Where:}
\begin{itemize}[nosep]
    \item $T_1$ = temperature at $r = r_1$ (\si{\kelvin})
    \item $T_2$ = temperature at $r = r_2$ (\si{\kelvin})
    \item $r_1$, $r_2$ = inner and outer radii (\si{\meter})
    \item $L$ = cylinder length (\si{\meter})
\end{itemize}

The associated \textbf{cylindrical conduction resistance} is

\begin{equation}
    R_{\text{cond,cyl}} = \frac{\ln(r_2 / r_1)}{2\pi k L}
    \label{eq:cylinder-R}
\end{equation}

\textbf{Purpose:} These expressions are used for pipes, insulation on pipes, and
other cylindrical geometries.

Similar expressions exist for spherical shells, with different area dependence,
but the same idea: conduction resistance depends on geometry.

\subsubsection*{Cylinders and spheres with convection on both sides}

For a \textbf{cylindrical shell} with inner radius $r_1$, outer radius $r_2$, length $L$, conductivity $k$, convection on the inner and outer surfaces with $h_1$, $h_2$, and bulk fluid temperatures $T_{\infty 1}$, $T_{\infty 2}$:

\begin{align}
    R_{\text{conv},1} &= \frac{1}{h_1 A_1} = \frac{1}{h_1 (2\pi r_1 L)} \\
    R_{\text{cond,cyl}} &= \frac{\ln(r_2 / r_1)}{2\pi k L} \\
    R_{\text{conv},2} &= \frac{1}{h_2 A_2} = \frac{1}{h_2 (2\pi r_2 L)}
\end{align}

Total resistance and heat transfer rate:

\begin{equation}
    R_{\text{total,cyl}} =
    R_{\text{conv},1} + R_{\text{cond,cyl}} + R_{\text{conv},2}
    \label{eq:cyl-total-R}
\end{equation}

\begin{equation}
    \dot{Q} =
    \frac{T_{\infty 1} - T_{\infty 2}}{R_{\text{total,cyl}}}
    \label{eq:cyl-total-Q}
\end{equation}

For a \textbf{spherical shell} with radii $r_1$, $r_2$:

\begin{align}
    A_1 &= 4\pi r_1^2, \quad A_2 = 4\pi r_2^2 \\
    R_{\text{cond,sph}} &= \frac{r_2 - r_1}{4\pi k r_1 r_2}
\end{align}

\begin{equation}
    R_{\text{total,sph}} =
    \frac{1}{h_1 A_1} + R_{\text{cond,sph}} + \frac{1}{h_2 A_2}
    \label{eq:sph-total-R}
\end{equation}

\begin{equation}
    \dot{Q} =
    \frac{T_{\infty 1} - T_{\infty 2}}{R_{\text{total,sph}}}
    \label{eq:sph-total-Q}
\end{equation}

Once $\dot{Q}$ is known, any surface or interface temperature $T_j$ is found from
\[
    \Delta T_{i\to j} = \dot{Q}\, R_{i\to j}
\]
just as in the plane-wall case.

\subsubsection*{Multilayer cylindrical and spherical shells}

For a \textbf{multilayer cylinder} with $n$ concentric layers (conductivities $k_i$, radii $r_i$):

\begin{equation}
    R_{\text{cond,total,cyl}} =
    \sum_{i=1}^{n}
    \frac{\ln(r_{i+1} / r_i)}{2\pi k_i L}
    \label{eq:multi-cyl-R}
\end{equation}

Total resistance with convection on both sides:

\begin{equation}
    R_{\text{total}} =
    \frac{1}{h_1 2\pi r_1 L}
    + \sum_{i=1}^{n} \frac{\ln(r_{i+1}/r_i)}{2\pi k_i L}
    + \frac{1}{h_2 2\pi r_{n+1} L}
    \label{eq:multi-cyl-Rtotal}
\end{equation}

An analogous expression holds for \textbf{multilayer spheres}, replacing each cylindrical conduction term with the spherical form

\[
    R_{\text{cond,sph},i} =
    \frac{r_{i+1} - r_i}{4\pi k_i r_i r_{i+1}}.
\]

\textbf{Purpose:} These formulas extend the resistance-network method to pipes with insulation, cladding, or multiple solid layers, and to spherical tanks.

%%%%%%%%%%%%%%%%%%%%%%%%%%%%%%%%%%%%%%%%%%%%%%%%%%%%%%%%%%%%
\subsection{17.6 Critical Radius of Insulation}

For \textbf{curved} geometries (cylinders, spheres), adding insulation has two competing effects:

\begin{itemize}[nosep]
    \item Increases conduction resistance (good for reducing heat transfer),
    \item Increases outer surface area $\to$ decreases convection resistance (could increase heat transfer).
\end{itemize}

For a cylinder of outer radius $r_1$ covered by insulation of conductivity $k$ up to radius $r_2$, exposed to convection coefficient $h$:

\begin{equation}
    R_{\text{ins}} = \frac{\ln(r_2 / r_1)}{2\pi k L},
    \qquad
    R_{\text{conv}} = \frac{1}{h (2\pi r_2 L)}.
\end{equation}

Total resistance:

\[
    R_{\text{total}}(r_2) = R_{\text{ins}} + R_{\text{conv}}.
\]

Differentiating $\dot{Q} = (T_1 - T_\infty)/R_{\text{total}}$ with respect to $r_2$ and setting $d\dot{Q}/dr_2 = 0$ gives the \textbf{critical radius}.

\subsubsection*{Critical radius for a cylinder}

\begin{equation}
    r_{\text{cr,cyl}} = \frac{k}{h}
    \label{eq:rcr-cylinder}
\end{equation}

\textbf{Interpretation:}
\begin{itemize}[nosep]
    \item If $r_2 < r_{\text{cr,cyl}}$: adding insulation \emph{increases} $\dot{Q}$ (heat loss goes up).
    \item If $r_2 = r_{\text{cr,cyl}}$: $\dot{Q}$ is maximized.
    \item If $r_2 > r_{\text{cr,cyl}}$: adding insulation behaves as expected and \emph{reduces} $\dot{Q}$.
\end{itemize}

For typical insulating materials ($k \approx 0.04$--$0.06\,\si{\watt\per\meter\per\kelvin}$) and natural convection in air ($h \gtrsim 5\,\si{\watt\per\meter\squared\per\kelvin}$), $r_{\text{cr,cyl}}$ is on the order of $\sim 1\,\si{\centi\meter}$ or less, so most pipe insulation is safely beyond the critical radius.

\subsubsection*{Critical radius for a sphere}

A similar derivation for a spherical shell yields

\begin{equation}
    r_{\text{cr,sph}} = \frac{2k}{h}
    \label{eq:rcr-sphere}
\end{equation}

\textbf{Purpose:} Critical radius analysis tells us whether adding insulation to small cylinders/spheres (e.g.\ thin wires) might actually \emph{increase} heat transfer (useful for electrical wires where we want better cooling), and assures us that for most practical pipe insulation the usual intuition ``more insulation = less heat loss'' holds.

%%%%%%%%%%%%%%%%%%%%%%%%%%%%%%%%%%%%%%%%%%%%%%%%%%%%%%%%%%%%
\subsection{17.7 Heat Transfer from Finned Surfaces}

Fins (extended surfaces) increase the area available for convection/radiation and can greatly enhance heat transfer from a base surface at $T_b$ to a fluid at $T_\infty$.

\subsubsection*{Governing differential equation for a fin}

Consider a straight fin of constant cross-section:
\begin{itemize}[nosep]
    \item Cross-sectional area $A_c$,
    \item Perimeter $p$,
    \item Thermal conductivity $k$,
    \item Convection coefficient $h$ to surroundings at $T_\infty$,
    \item Temperature excess $u(x) = T(x) - T_\infty$.
\end{itemize}

Energy balance on a differential element gives the \textbf{fin equation}:

\begin{equation}
    \frac{d^2 u}{dx^2} - m^2 u = 0,
    \qquad
    m^2 = \frac{h p}{k A_c}
    \label{eq:fin-ode}
\end{equation}

General solution:
\[
    u(x) = C_1 e^{m x} + C_2 e^{-m x}.
\]
Constants are set by boundary conditions at the fin base and tip.

\subsubsection*{Boundary condition at the base}

At $x = 0$ the fin is attached to a wall at known temperature $T_b$:
\begin{equation}
    u(0) = u_b = T_b - T_\infty
    \label{eq:fin-bc-base}
\end{equation}

\subsubsection*{Common fin-tip conditions and solutions}

\paragraph{1. Infinitely long fin} ($T(L)\to T_\infty$, or $u(L)\to 0$ as $L\to\infty$):

\begin{align}
    \frac{T(x) - T_\infty}{T_b - T_\infty} &= e^{-m x}
    \label{eq:fin-long-T} \\
    \dot{Q}_{\text{long fin}} &= \sqrt{h p k A_c}\,(T_b - T_\infty)
    \label{eq:fin-long-Q}
\end{align}

\paragraph{2. Negligible heat loss at tip} (adiabatic tip, $dT/dx|_{x=L} = 0$):

\begin{align}
    \frac{T(x) - T_\infty}{T_b - T_\infty}
        &= \frac{\cosh\bigl(m(L - x)\bigr)}{\cosh(mL)}
        \label{eq:fin-adiabatic-T} \\
    \dot{Q}_{\text{adiabatic tip}}
        &= \sqrt{h p k A_c}\,(T_b - T_\infty)\,\tanh(mL)
        \label{eq:fin-adiabatic-Q}
\end{align}

\paragraph{3. Specified tip temperature} $T(L) = T_L$:

\begin{equation}
    \frac{T(x) - T_\infty}{T_b - T_\infty}
    = \frac{\displaystyle
    \frac{T_L - T_\infty}{T_b - T_\infty} \sinh(m x)
    + \sinh\bigl(m(L - x)\bigr)}
    {\sinh(mL)}
    \label{eq:fin-specified-T}
\end{equation}

\paragraph{4. Convection at tip} (with same $h$):

Tip condition from energy balance:
\[
    -k A_c \frac{dT}{dx}\bigg|_{x=L}
    = h A_c \bigl(T(L) - T_\infty\bigr).
\]

Solution for temperature distribution:

\begin{equation}
    \frac{T(x) - T_\infty}{T_b - T_\infty}
    =
    \frac{
        \cosh\bigl(m(L - x)\bigr)
        + \frac{h}{m k} \sinh\bigl(m(L - x)\bigr)
    }{
        \cosh(mL) + \frac{h}{m k} \sinh(mL)
    }
    \label{eq:fin-conv-T}
\end{equation}

Corresponding heat transfer rate:

\begin{equation}
    \dot{Q}_{\text{conv tip}} =
    \sqrt{h p k A_c}\,(T_b - T_\infty)\,
    \frac{\sinh(mL) + \frac{h}{m k}\cosh(mL)}
         {\cosh(mL) + \frac{h}{m k}\sinh(mL)}
    \label{eq:fin-conv-Q}
\end{equation}

\subsubsection*{Corrected fin length}

In practice, fins with convection at the tip are often approximated as \emph{adiabatic-tip} fins by using a \textbf{corrected length} $L_c$:

\begin{equation}
    L_c = L + \frac{A_c}{p}
    \label{eq:Lc-general}
\end{equation}

Common cases:
\begin{align}
    &\text{Rectangular fin (thickness $t$)}: \quad L_c = L + \frac{t}{2} \\
    &\text{Cylindrical pin fin (diameter $D$)}: \quad L_c = L + \frac{D}{4}
\end{align}

Then use the adiabatic-tip formulas \eqref{eq:fin-adiabatic-T}--\eqref{eq:fin-adiabatic-Q} with $L$ replaced by $L_c$.

\textbf{Purpose:} Simplifies analysis while giving accurate results when $mL \gtrsim 1$.

\subsubsection*{Fin efficiency}

Define the \emph{fin efficiency} $\eta_{\text{fin}}$ as

\begin{equation}
    \eta_{\text{fin}} =
    \frac{\dot{Q}_{\text{fin}}}{\dot{Q}_{\text{fin,max}}}
    =
    \frac{\text{actual heat transfer from fin}}
         {h A_{\text{fin}} (T_b - T_\infty)}
    \label{eq:eta-fin-def}
\end{equation}

For a straight fin of constant cross section:

\begin{itemize}[nosep]
    \item \textbf{Very long fin:}
    \begin{equation}
        \eta_{\text{long}} =
        \frac{1}{mL}
        \label{eq:eta-long}
    \end{equation}

    \item \textbf{Adiabatic-tip fin (or convective tip with corrected $L_c$):}
    \begin{equation}
        \eta_{\text{adiabatic}} =
        \frac{\tanh(mL)}{mL}
        \label{eq:eta-adiabatic}
    \end{equation}
\end{itemize}

Fin heat transfer in terms of efficiency:

\begin{equation}
    \dot{Q}_{\text{fin}} =
    \eta_{\text{fin}}\, h A_{\text{fin}} (T_b - T_\infty)
    \label{eq:Q-fin-eta}
\end{equation}

\textbf{Rule of thumb:} fins with $\eta_{\text{fin}} \lesssim 0.6$ are usually not economical (too long / too much material for little extra heat transfer).

\subsubsection*{Fin effectiveness}

\textbf{Fin effectiveness} measures how worthwhile it is to add the fin compared to leaving just the base area $A_b$ exposed:

\begin{equation}
    \varepsilon_{\text{fin}} =
    \frac{\dot{Q}_{\text{fin}}}{\dot{Q}_{\text{no fin}}}
    = \frac{\dot{Q}_{\text{fin}}}{h A_b (T_b - T_\infty)}
    = \frac{A_{\text{fin}}}{A_b}\,\eta_{\text{fin}}
    \label{eq:fin-effectiveness}
\end{equation}

Interpretation:
\begin{itemize}[nosep]
    \item $\varepsilon_{\text{fin}} < 1$: fin acts like \emph{insulation} (bad design).
    \item $\varepsilon_{\text{fin}} \approx 1$: fin pointless (little benefit).
    \item $\varepsilon_{\text{fin}} \gg 1$: fin significantly enhances heat transfer (desirable).
\end{itemize}

\textbf{Design implications:}
\begin{itemize}[nosep]
    \item Want high $k$ (good conductor: aluminum, copper),
    \item Want large $p/A_c$ (thin, slender fins),
    \item Fins are most effective at low $h$ (gases, natural convection) and on the gas side of heat exchangers.
\end{itemize}

\subsubsection*{Overall effectiveness of a finned surface}

For a surface with unfinned area $A_{\text{unfin}}$ and total fin area $A_{\text{fin}}$ (sum of all fins), and base temperature $T_b$:

\begin{align}
    \dot{Q}_{\text{total, finned}}
        &= h A_{\text{unfin}} (T_b - T_\infty)
        + \eta_{\text{fin}} h A_{\text{fin}} (T_b - T_\infty) \\
        &= h (A_{\text{unfin}} + \eta_{\text{fin}} A_{\text{fin}})(T_b - T_\infty)
        \label{eq:Q-total-finned}
\end{align}

If the same geometric area with no fins is $A_{\text{no fin}}$, define \textbf{overall surface effectiveness}:

\begin{equation}
    \varepsilon_{\text{overall}} =
    \frac{\dot{Q}_{\text{total, finned}}}{\dot{Q}_{\text{total, no fin}}}
    =
    \frac{A_{\text{unfin}} + \eta_{\text{fin}} A_{\text{fin}}}{A_{\text{no fin}}}
    \label{eq:overall-effectiveness}
\end{equation}

This accounts for both the individual fin performance and how many fins are packed onto the surface.

\subsubsection*{Thermal resistance of a heat sink}

Commercial heat sinks (for electronics) are often characterized by an overall thermal resistance between the \emph{base} (temperature $T_b$) and ambient ($T_\infty$):

\begin{equation}
    R_{\text{hs}} = \frac{T_b - T_\infty}{\dot{Q}_{\text{hs}}}
    \quad[\si{\kelvin\per\watt}]
    \label{eq:R-heat-sink}
\end{equation}

\textbf{Purpose:} This single number captures the combined effect of fin geometry, material, and natural/forced convection, and is used directly to check whether a heat sink keeps an electronic component below its maximum safe temperature.

%===========================
% Chapter 18 – Transient Conduction
%===========================

\section{Chapter 18: Transient Heat Conduction}

\subsection{18.1 Transient Heat Conduction \& Lumped System Analysis}

In transient (unsteady) conduction, the temperature depends on both position and time:
\[
T = T(x,y,z,t).
\]
We are usually interested in how a solid heats up or cools down when its surroundings change suddenly (step change in $T_\infty$ or $h$).

\subsubsection*{Energy balance for a solid}

For a solid of volume $V$ and surface area $A_s$, with uniform temperature $T(t)$ (no internal generation), exchanging heat with a fluid at $T_\infty$ by convection:
\[
\dot{Q}_{\text{conv}} = h A_s \left[T(t) - T_\infty\right],
\]
and the rate of change of internal energy is
\[
\dot{E}_{\text{stored}} = \rho V c_p \,\frac{\mathrm{d}T}{\mathrm{d}t}.
\]

Energy conservation (cooling case, $T > T_\infty$):
\[
-\dot{Q}_{\text{conv}} = \dot{E}_{\text{stored}}
\quad\Rightarrow\quad
- h A_s \left[T(t) - T_\infty\right] = \rho V c_p\,\frac{\mathrm{d}T}{\mathrm{d}t}.
\]

This first-order ODE has the solution
\[
T(t) - T_\infty = \left(T_i - T_\infty\right)\exp\!\left(-\frac{hA_s}{\rho V c_p}t\right),
\]
or in dimensionless form
\[
\theta(t) \equiv \frac{T(t)-T_\infty}{T_i - T_\infty}
    = \exp\!\left(-\frac{hA_s}{\rho V c_p}t\right).
\]

\paragraph{Variables.}
\begin{itemize}
  \item $T(t)$: uniform temperature of the solid at time $t$.
  \item $T_\infty$: ambient (free-stream) fluid temperature.
  \item $T_i$: initial temperature of the solid at $t = 0$.
  \item $h$: convection heat transfer coefficient.
  \item $A_s$: surface area of the solid.
  \item $\rho$: density of the solid.
  \item $c_p$: specific heat of the solid.
\end{itemize}

\subsubsection*{Characteristic length, Biot number, Fourier number}

For conduction problems we define a characteristic length
\[
L_c = \frac{V}{A_s},
\]
and the \textbf{thermal diffusivity}
\[
\alpha = \frac{k}{\rho c_p}.
\]

Two important dimensionless groups:

\[
\mathrm{Bi} = \frac{h L_c}{k}
\quad\text{(Biot number)}
\]

\[
\mathrm{Fo} = \frac{\alpha t}{L_c^2}
\quad\text{(Fourier number)}.
\]

\paragraph{Physical meaning.}
\begin{itemize}
  \item $\mathrm{Bi}$ compares \emph{internal conduction resistance} in the solid to \emph{external convection resistance}. 
        \begin{itemize}
          \item $\mathrm{Bi} \ll 1$ (typically $\mathrm{Bi} \le 0.1$): conduction inside is very fast compared to convection; the solid tends to remain nearly uniform in temperature (lumped system is valid).
          \item $\mathrm{Bi} \gtrsim 0.1$: significant temperature gradients inside the solid; lumped analysis is no longer accurate.
        \end{itemize}
  \item $\mathrm{Fo}$ measures \emph{how long} heat has had to diffuse relative to the size of the object. Larger $\mathrm{Fo}$ means more time for temperature to “even out”.
\end{itemize}

\subsubsection*{Lumped-system solution in dimensionless form}

Using $L_c = V/A_s$ and $\alpha = k/(\rho c_p)$, the lumped solution can be written as
\[
\theta(t) = \frac{T(t)-T_\infty}{T_i - T_\infty}
           = \exp\!\left(-\mathrm{Bi}\,\mathrm{Fo}\right),
\]
with
\[
\mathrm{Bi} = \frac{h L_c}{k}, 
\qquad
\mathrm{Fo} = \frac{\alpha t}{L_c^2}.
\]

\paragraph{When is lumped analysis valid?}
\begin{itemize}
  \item Compute $\mathrm{Bi} = h L_c / k$ using the \emph{solid}'s properties and characteristic length.
  \item If $\mathrm{Bi} \le 0.1$, the lumped model is usually acceptable.
  \item If $\mathrm{Bi} > 0.1$, internal gradients are important and a non-lumped (spatially varying) solution is needed.
\end{itemize}


\subsection{18.2 One- and Multi-Dimensional Non-Lumped Transient Conduction}

When $\mathrm{Bi}$ is not small, temperature varies inside the solid. The governing equation (no generation) for 1-D transient conduction in a plane wall is
\[
\frac{\partial^2 T}{\partial x^2}
  = \frac{1}{\alpha}\,\frac{\partial T}{\partial t},
\]
with appropriate boundary and initial conditions (e.g.\ symmetry at the center, convection at the surface, and uniform initial temperature).

\subsubsection*{Non-dimensional formulation (plane wall example)}

Define
\[
X = \frac{x}{L}, 
\qquad
u(X,t) = \frac{T(x,t)-T_\infty}{T_i - T_\infty},
\qquad
t = \mathrm{Fo} = \frac{\alpha t_{\text{phys}}}{L^2}.
\]

Then the 1-D transient conduction problem in a plane wall can be written in dimensionless form as
\begin{align*}
\frac{\partial^2 u}{\partial X^2} &= \frac{\partial u}{\partial t}, \\
\left.\frac{\partial u}{\partial X}\right|_{X=0} &= 0, \\
\left.\frac{\partial u}{\partial X}\right|_{X=1} &= -\mathrm{Bi}\,u(1,t), \\
u(X,0) &= 1.
\end{align*}

The solution for $u(X,t)$ is an infinite series involving eigenvalues that depend on $\mathrm{Bi}$. In practice, we usually:
\begin{itemize}
  \item Use \textbf{Heisler charts} or tables for $u(X,t)$ as a function of $X$, $\mathrm{Bi}$, and $\mathrm{Fo}$.
  \item Or use a \textbf{one-term approximation} (first term of the series) once time is not too small, e.g.\ for $\mathrm{Fo} \gtrsim 0.2$.
\end{itemize}

\paragraph{General form of the 1-D solution.}
For a plane wall with convection at the surface, the dimensionless temperature often has the form
\[
u(X,t) \approx A_1 \exp\!\left(-\lambda_1^2 t\right)\,\phi_1(X),
\]
where $\lambda_1$ and $A_1$ depend on $\mathrm{Bi}$ (tabulated), and $\phi_1(X)$ is a spatial eigenfunction (e.g.\ cosine or Bessel function depending on geometry).

Similar formulations exist for:
\begin{itemize}
  \item \textbf{Long cylinder}: conduction in $r$ only.
  \item \textbf{Sphere}: conduction in $r$ only.
\end{itemize}

\subsubsection*{Multidimensional problems and product solution}

Near corners and edges, heat can flow in more than one direction (e.g.\ $x$ and $y$). For some geometries (rectangular plates, etc.) with simple boundary conditions, the \textbf{product solution} approach can be used:
\[
u(x,y,t) = u_x(x,t)\,u_y(y,t),
\]
or more generally as a sum of such products. Each factor satisfies a 1-D transient conduction problem in its coordinate direction.

Key ideas:
\begin{itemize}
  \item Multidimensional solutions are built from 1-D eigenfunctions in each coordinate direction.
  \item In practice, engineers often use charts/tables or numerical methods (software) for true 2-D/3-D transients.
\end{itemize}


\subsection{18.3 Semi-Infinite Body}

A \textbf{semi-infinite medium} occupies $x \ge 0$ and is so large that the disturbance at the surface never reaches the “far” end (we treat it as infinite).

Typical setup:
\begin{itemize}
  \item Initial condition: $T(x,0) = T_i$ for $x \ge 0$.
  \item Boundary condition at the surface: $T(0,t) = T_s$ (sudden change) or convection at $x=0$.
  \item As $x \to \infty$: $T(x,t) \to T_i$.
\end{itemize}

\subsubsection*{Similarity variable}

For the constant surface-temperature case, we introduce the \textbf{similarity variable}
\[
\eta = \frac{x}{2\sqrt{\alpha t}},
\]
which collapses $x$ and $t$ into a single variable. The dimensionless temperature
\[
\theta(x,t) = \frac{T(x,t) - T_s}{T_i - T_s}
\]
depends only on $\eta$, not on $x$ and $t$ separately:
\[
\theta = \theta(\eta).
\]

The analytical solution is expressed in terms of the (Gaussian) \textbf{error function}:
\[
\theta(\eta) = \operatorname{erf}(\eta)
\quad\Rightarrow\quad
\frac{T(x,t)-T_s}{T_i - T_s} = \operatorname{erf}\!\left(\frac{x}{2\sqrt{\alpha t}}\right).
\]

\paragraph{Physical interpretation.}
\begin{itemize}
  \item For small $t$, only a thin layer near the surface has responded to the temperature change.
  \item The \textbf{thermal penetration depth} grows roughly like
        \[
        \delta \sim \mathcal{O}\big(\sqrt{\alpha t}\big).
        \]
  \item For $x \gg \sqrt{\alpha t}$, the temperature is still essentially $T_i$.
\end{itemize}


%===========================
% Chapter 19 – Forced Convection
%===========================

\section{Chapter 19: Forced Convection and Boundary Layers}

\subsection{19.1 Forced Convection and the Velocity Boundary Layer}

\textbf{Forced convection} occurs when fluid motion is driven by a fan, pump, blower, or external flow (e.g.\ air flowing over a plate). The convective heat transfer rate is given by Newton’s law of cooling:
\[
\dot{Q} = h A_s (T_s - T_\infty).
\]

When a fluid flows over a solid surface:
\begin{itemize}
  \item At the surface: due to the no-slip condition, $u = 0$.
  \item Away from the surface: $u \to U_\infty$ (free-stream velocity).
\end{itemize}
The region where the velocity changes from $0$ to $U_\infty$ is the \textbf{velocity boundary layer}.

\paragraph{Velocity boundary layer thickness.}
The \textbf{boundary layer thickness} $\delta(x)$ is typically defined where $u \approx 0.99\,U_\infty$. It:
\begin{itemize}
  \item Starts at zero at the leading edge ($x=0$).
  \item Grows downstream as the fluid is slowed by viscous effects.
  \item Is thinner in high-speed or low-viscosity flows (large Reynolds number).
\end{itemize}


\subsection{19.2 Thermal Boundary Layer}

Similarly, there is a \textbf{thermal boundary layer}, where the fluid temperature changes from $T_s$ at the surface toward $T_\infty$ in the free stream.

\begin{itemize}
  \item The thickness of the thermal boundary layer is denoted $\delta_t(x)$.
  \item The relative thicknesses of $\delta_t$ and $\delta$ depend on the \textbf{Prandtl number}.
\end{itemize}

\paragraph{Prandtl number.}
\[
\mathrm{Pr} = \frac{\nu}{\alpha} = \frac{\mu c_p}{k},
\]
where $\nu$ is kinematic viscosity and $\alpha$ is thermal diffusivity.

\begin{itemize}
  \item $\mathrm{Pr} \ll 1$ (liquid metals): thermal diffusion is very fast; $\delta_t$ is \emph{thicker} than the velocity boundary layer.
  \item $\mathrm{Pr} \gg 1$ (oils): thermal diffusion is slow; $\delta_t$ is \emph{thinner} than the velocity boundary layer.
  \item $\mathrm{Pr} \approx 1$ (many gases, water at some conditions): $\delta_t$ and $\delta$ are of similar thickness.
\end{itemize}


\subsection{19.3 Reynolds, Prandtl, and Nusselt Numbers}

\paragraph{Reynolds number (external flow over flat plate).}
For a position $x$ from the leading edge:
\[
\mathrm{Re}_x = \frac{\rho U_\infty x}{\mu} = \frac{U_\infty x}{\nu}.
\]

\begin{itemize}
  \item Small $\mathrm{Re}_x$: viscous forces dominate, flow is laminar.
  \item Large $\mathrm{Re}_x$: inertial forces dominate, flow becomes turbulent.
  \item For a flat plate, transition typically begins around $\mathrm{Re}_x \sim 10^5$ and is often taken as fully turbulent after $\mathrm{Re}_x \sim 3\times10^6$. A commonly used \emph{engineering} critical value is
        \[
        \mathrm{Re}_{\text{cr}} \approx 5\times 10^5.
        \]
\end{itemize}

\paragraph{Prandtl number (again).}
As above,
\[
\mathrm{Pr} = \frac{\nu}{\alpha} = \frac{\mu c_p}{k}.
\]
It compares momentum and thermal diffusivities and controls the relative thickness of velocity and thermal boundary layers.

\paragraph{Nusselt number.}
The \textbf{Nusselt number} is a dimensionless heat transfer coefficient:
\[
\mathrm{Nu} = \frac{h L_c}{k}.
\]
For local values over a flat plate we use
\[
\mathrm{Nu}_x = \frac{h_x x}{k},
\]
and for average over a plate of length $L$,
\[
\mathrm{Nu}_L = \frac{h_L L}{k}.
\]

Physical meaning:
\begin{itemize}
  \item $\mathrm{Nu}$ measures the enhancement of heat transfer by convection relative to pure conduction across the same distance.
  \item Larger $\mathrm{Nu}$ means stronger convective transport and a larger $h$.
\end{itemize}


\subsection{19.4 Flow over Flat Plates, Cylinders, and Spheres}

\subsubsection*{Flow over a flat plate (isothermal surface)}

For a flat plate at uniform surface temperature $T_s$ in a fluid at $T_\infty$, with free-stream velocity $U_\infty$:

\paragraph{Local Nusselt number (laminar).}
\[
\mathrm{Nu}_x = \frac{h_x x}{k}
  = 0.332\,\mathrm{Re}_x^{1/2}\,\mathrm{Pr}^{1/3},
\quad
\mathrm{Pr} > 0.6,\;
\mathrm{Re}_x < 5\times 10^5.
\]

\paragraph{Local Nusselt number (turbulent).}
\[
\mathrm{Nu}_x = 0.0296\,\mathrm{Re}_x^{0.8}\,\mathrm{Pr}^{1/3},
\quad
0.6 \le \mathrm{Pr} \le 60,\;
5\times 10^5 \lesssim \mathrm{Re}_x \lesssim 10^7.
\]

\paragraph{Average Nusselt number over the plate.}
For a plate of length $L$:
\[
\mathrm{Nu}_L = \frac{h_L L}{k}.
\]

If the flow is laminar over the entire plate:
\[
\mathrm{Nu}_L = 0.664\,\mathrm{Re}_L^{1/2}\,\mathrm{Pr}^{1/3},
\quad
\mathrm{Re}_L < 5\times 10^5.
\]

If the flow is turbulent over the entire plate:
\[
\mathrm{Nu}_L = 0.037\,\mathrm{Re}_L^{0.8}\,\mathrm{Pr}^{1/3},
\quad
0.6 \le \mathrm{Pr} \le 60,\;
5\times 10^5 \lesssim \mathrm{Re}_L \lesssim 10^7.
\]

For cases where the flow is laminar near the leading edge and turbulent further downstream, a combined-correlation often used in practice is
\[
\mathrm{Nu}_L = \left(0.037\,\mathrm{Re}_L^{0.8} - 871\right)\mathrm{Pr}^{1/3},
\quad
0.6 \le \mathrm{Pr} \le 60.
\]

\subsubsection*{Cross-flow over cylinders and spheres}

For external flow over a cylinder or sphere (diameter $D$), we use $D$ as the characteristic length, so
\[
\mathrm{Re} = \frac{\rho U_\infty D}{\mu}, \qquad
\mathrm{Nu} = \frac{h D}{k}.
\]

\paragraph{Cylinder (cross-flow).}
A widely used correlation:
\[
\mathrm{Nu}_{\text{cyl}} =
0.3 + 0.62\,\mathrm{Re}^{1/2}\mathrm{Pr}^{1/3}
\left[1 + \left(\frac{0.4}{\mathrm{Pr}}\right)^{2/3}\right]^{1/4}
\left[1 + \left(\frac{\mathrm{Re}}{282{,}000}\right)^{5/8}\right]^{4/5},
\]
valid for $\mathrm{Re}\,\mathrm{Pr} \gtrsim 0.2$ (typical external flows).

\paragraph{Sphere (cross-flow).}
A commonly used correlation:
\[
\mathrm{Nu}_{\text{sph}} =
2 + \left[0.4\,\mathrm{Re}^{1/2} + 0.06\,\mathrm{Re}^{2/3}\right]
\mathrm{Pr}^{0.4}\left(\frac{\mu_\infty}{\mu_s}\right)^{1/4},
\]
for about $3.5 \le \mathrm{Re} \le 8\times 10^4$ and $0.7 \le \mathrm{Pr} \le 380$. Here $\mu_\infty$ is evaluated at the free-stream temperature, and $\mu_s$ at the surface temperature.

\paragraph{How to use these correlations in practice.}
\begin{enumerate}
  \item Choose the characteristic length:
        \begin{itemize}
          \item Flat plate: $L$ in flow direction (for $\mathrm{Nu}_L$).
          \item Cylinder or sphere: diameter $D$.
        \end{itemize}
  \item Compute $\mathrm{Re}$ and $\mathrm{Pr}$ using fluid properties at the \emph{film temperature} (typically $T_f = (T_s + T_\infty)/2$ for external flow).
  \item Select the correct regime (laminar / turbulent / combined).
  \item Compute $\mathrm{Nu}$, then solve for $h$ from
        \[
        h = \frac{\mathrm{Nu}\,k}{L_c}.
        \]
  \item Finally, compute $\dot{Q} = h A_s (T_s - T_\infty)$ as needed.
\end{enumerate}

%=========================================
% Chapter 21 – Thermal Radiation
%=========================================

\section{Chapter 21: Thermal Radiation}

Thermal radiation is energy emitted by matter due to its temperature, in the form of electromagnetic waves. 
Unlike conduction and convection, radiation:
\begin{itemize}[nosep]
    \item requires no medium (can occur in a vacuum),
    \item travels at the speed of light,
    \item depends strongly on surface properties and temperature,
    \item is governed by electromagnetic wave physics rather than fluid mechanics.
\end{itemize}

Radiation becomes especially significant at high temperatures because radiative heat flux scales with $T^4$.


%-----------------------------------------
\subsection{21.1 Fundamentals of Thermal Radiation}
%-----------------------------------------

All matter at $T > 0\,\mathrm{K}$ emits thermal radiation. The emission originates from oscillations of electrons in atoms/molecules.

The key quantity is the \textbf{spectral emissive power}, $E_\lambda(T)$:
\[
E_\lambda(T) = \text{energy emitted per unit area per unit wavelength per unit time}.
\]

The \textbf{total emissive power} (all wavelengths) is:
\[
E(T) = \int_0^\infty E_\lambda(T)\,\mathrm{d}\lambda.
\]

A theoretical surface that emits the maximum possible radiation at any wavelength and temperature is a \textbf{blackbody}. It serves as the reference for all real surfaces.

\subsubsection*{Stefan–Boltzmann law}
The total radiation emitted by a blackbody is
\[
E_b = \sigma T^4,
\qquad
\sigma = 5.67 \times 10^{-8}\,\mathrm{W/m^2\cdot K^4}.
\]

Real surfaces emit less than a blackbody. Their emissive power is
\[
E = \varepsilon \sigma T^4,
\]
where $0 \le \varepsilon \le 1$ is the \textbf{emissivity}.

\subsubsection*{Kirchhoff’s law (important)}
For a surface in thermal equilibrium:
\[
\varepsilon_\lambda = \alpha_\lambda.
\]
That is, a good absorber is a good emitter at the same wavelength.

This becomes essential when studying radiation exchange between surfaces.


%-----------------------------------------
\subsection*{(Very Brief) Essentials from 21.2–21.3}
%-----------------------------------------

Although not covered in lecture, we need two basic ideas to understand later sections:

\paragraph{1. Spectral radiative properties.}
Surfaces interact with radiation via:
\[
\alpha_\lambda + \rho_\lambda + \tau_\lambda = 1
\]
(absorption + reflection + transmission).

Opaque surfaces: $\tau_\lambda = 0$.

\paragraph{2. Directional radiative behavior.}
Real surfaces may emit or reflect differently depending on direction. A \textbf{diffuse} surface emits uniformly in all directions. Most engineering surfaces are assumed diffuse for simplicity.

These basics enable the radiation network methods used in Sections 21.5–21.8.


%-----------------------------------------
\subsection{21.4 Radiative Properties of Surfaces}
%-----------------------------------------

A surface interacting with incident radiation has:

\begin{itemize}[nosep]
    \item \textbf{Absorptivity} $\alpha$: fraction absorbed.
    \item \textbf{Reflectivity} $\rho$: fraction reflected.
    \item \textbf{Transmissivity} $\tau$: fraction transmitted (usually 0 for solids).
\end{itemize}

For opaque solids:
\[
\alpha + \rho = 1.
\]

\subsubsection*{Emissivity}
\[
\varepsilon = \frac{\text{actual emitted radiation}}{\text{blackbody emission at same }T}.
\]

A ``black'' surface has $\varepsilon = 1$.  
Polished metals have very low emissivity ($0.02$–$0.1$), and rough/dark surfaces have high emissivity ($0.8$–$0.95$).

\subsubsection*{Gray surface approximation}
If $\varepsilon$ and $\alpha$ do not vary significantly with wavelength, we treat the surface as \textbf{gray}:
\[
\varepsilon = \text{constant over all wavelengths}.
\]

This dramatically simplifies radiation heat transfer.


%-----------------------------------------
\subsection{21.5 View Factors}
%-----------------------------------------

When two surfaces exchange radiation, not all energy leaving surface 1 reaches surface 2. The geometric fraction that does is the \textbf{view factor} (or configuration factor):
\[
F_{1\to2}
= \text{fraction of radiation leaving surface 1 that strikes surface 2 directly}.
\]

It depends only on geometry, not temperature or material.

\subsubsection*{View factor definition}
\[
F_{1\to2}
= \frac{1}{A_1}
\int_{A_1}\int_{A_2}
\frac{\cos\theta_1 \cos\theta_2}{\pi r^2}\, \mathrm{d}A_1\,\mathrm{d}A_2.
\]

In practice we rarely evaluate this integral; instead we use known formulas or view factor algebra.

\subsubsection*{Important view factor rules}

\paragraph{1. Reciprocity relation}
\[
A_1 F_{1\to2} = A_2 F_{2\to1}.
\]

\paragraph{2. Summation rule}
For any enclosure:
\[
\sum_{j=1}^N F_{i\to j} = 1.
\]

\paragraph{3. Surface to itself}
\[
F_{i\to i} = 0
\quad\text{for planar surfaces.}
\]

\subsubsection*{Common special cases}
\begin{itemize}[nosep]
    \item A large plane facing a small object: $F_{obj \to wall} = 1$ but $F_{wall \to obj} \ll 1$ from reciprocity.
    \item Two infinite parallel plates: $F_{1\to2} = 1$.
    \item Small surface inside a large enclosure: $F_{1\to surroundings} = 1$.
\end{itemize}

These will be crucial for black and gray surface radiation exchange.


%-----------------------------------------
\subsection{21.6 Radiation Exchange Between Black Surfaces}
%-----------------------------------------

A black surface emits radiation as $\sigma T^4$ and absorbs \emph{all} incident radiation ($\alpha = 1$).  
Therefore blackbody radiation exchange between two surfaces is the simplest case.

For two diffuse, black surfaces 1 and 2:
\[
\dot{Q}_{1\to2} = A_1 F_{1\to2}\,\sigma \left(T_1^4 - T_2^4\right).
\]

This result is extremely useful:
\begin{itemize}[nosep]
    \item No radiosity or iterative calculation.
    \item No emissivity terms.
    \item Only geometry (via view factors) and temperatures matter.
\end{itemize}

\subsubsection*{Enclosures with more surfaces}
In an enclosure with multiple black surfaces, net heat loss from surface $i$ is:
\[
\dot{Q}_i = \sum_{j=1}^N A_i F_{i\to j} \sigma \left(T_i^4 - T_j^4\right).
\]

Black-surface analysis forms the conceptual foundation for gray-surface networks.


%-----------------------------------------
\subsection{21.7 Radiation Networks and Resistances (Black Surfaces)}
%-----------------------------------------

Although black surfaces have the simplest exchange relationships, the \textbf{network method} helps visualize and generalize to gray surfaces.

\subsubsection*{Radiation network concept}
We define a “radiation potential”
\[
E_b = \sigma T^4,
\]
analogous to voltage.  
The resistance between two black surfaces is:
\[
R_{1\to2} = \frac{1}{A_1 F_{1\to2}}.
\]

Thus:
\[
\dot{Q}_{1\to2} = \frac{E_{b1} - E_{b2}}{R_{1\to2}}.
\]

This matches your convection-conduction electrical analogy from earlier chapters.

The radiation network becomes essential when emissivity is not 1.


%-----------------------------------------
\subsection{21.8 Radiation Exchange Between Gray, Diffuse Surfaces}
%-----------------------------------------

Real surfaces have emissivity $\varepsilon < 1$ and may reflect radiation.  
We use \textbf{radiosity} and \textbf{irradiation}:

\begin{itemize}[nosep]
    \item Radiosity $J$: total radiation \emph{leaving} a surface (emitted + reflected).
    \item Irradiation $G$: total radiation \emph{arriving at} a surface.
\end{itemize}

\subsubsection*{21.8.1 Surface relations}

For a gray, diffuse surface:
\[
J = \varepsilon E_b + (1 - \varepsilon) G.
\]

Net heat leaving the surface:
\[
\dot{q} = J - G.
\]

\subsubsection*{Convective-like radiation coefficients}
We define a \textbf{surface radiation resistance}:
\[
R_{\text{surf},i} = \frac{1 - \varepsilon_i}{\varepsilon_i A_i}.
\]

And an \textbf{exchange resistance} (from view factor):
\[
R_{i\to j} = \frac{1}{A_i F_{i\to j}}.
\]

Together these form a thermal-resistance-like network similar to conduction and convection.

\subsubsection*{21.8.2 Two gray surfaces}

For two diffuse, gray surfaces exchanging radiation:
\[
\dot{Q} = \frac{E_{b1} - E_{b2}}{
    R_{\text{surf},1} + R_{1\to 2} + R_{\text{surf},2}
}.
\]

Explicitly:
\[
\dot{Q} = 
\frac{\sigma(T_1^4 - T_2^4)}{
    \frac{1-\varepsilon_1}{\varepsilon_1 A_1}
    + \frac{1}{A_1 F_{1\to2}}
    + \frac{1-\varepsilon_2}{\varepsilon_2 A_2}
}.
\]

\paragraph{If the two surfaces form an enclosure (common case)}
Then $F_{1\to2} = 1$ and $A_1 = A_2$:
\[
\dot{Q}
= \frac{\sigma A \left(T_1^4 - T_2^4\right)}{
    \frac{1}{\varepsilon_1} + \frac{1}{\varepsilon_2} - 1
}.
\]

This is one of the most commonly used gray-surface formulas in engineering.


\subsubsection*{21.8.3 Radiation Exchange With Many Gray Surfaces}

The full multidimensional problem uses:
\begin{itemize}[nosep]
    \item A radiosity node at each surface,
    \item A surface resistance $R_{\text{surf},i}$,
    \item A geometry resistance $R_{i\to j}$,
    \item Energy conservation at each node.
\end{itemize}

This is analogous to solving a linear circuit with multiple nodes.  
However, for most practical textbook problems, only two-surface gray radiation is required.

\bigskip

\textbf{Summary of radiation exchange steps (gray surfaces):}
\begin{enumerate}[nosep]
    \item Compute $E_{bi} = \sigma T_i^4$.
    \item Compute $R_{\text{surf},i} = \dfrac{1-\varepsilon_i}{\varepsilon_i A_i}$.
    \item Compute $R_{1\to2} = \dfrac{1}{A_1 F_{1\to 2}}$.
    \item Use 
    \[
    \dot{Q} = \dfrac{E_{b1} - E_{b2}}{
        R_{\text{surf},1} + R_{1\to2} + R_{\text{surf},2}
    }.
    \]
\end{enumerate}

This parallels conduction and convection resistance networks and matches the style of earlier chapters.

\end{document}
